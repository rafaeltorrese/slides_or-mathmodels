\documentclass{beamer}

\usepackage[spanish,es-tabla]{babel}
\decimalpoint
\usepackage[utf8]{inputenc}
\usepackage{ragged2e} % este paquete es para justificar.
\justifying
\usepackage{booktabs}
\usepackage{colortbl}
\usepackage{verbatim}
\usepackage{amsmath}
\usepackage{bm} % bold font greek letters. Use \boldsymbol{ }  and \pmb
\usepackage{xfrac} % for slanted fractions   \sfrac{}{}
\usepackage{nicefrac} % for small fractions in text or math mode   \nicefrac{}{}
%\usepackage{blkarray} % for block arrays, useful for markov chains. Error with Metropolis Beamer Theme
\usepackage{hhline} % hlines  but interacts with vertical lines
\usepackage{multirow}
\usepackage{pgfpages} % for handouts
\usepackage{graphicx} % for graphics and grapics path
\usepackage{bm} % math bold symbols \bm{}
% \usepackage{blkarray} % for block arrays, useful for markov chain (matrices)
\usepackage{cancel}
\usepackage{systeme}
%\usepackage{enumitem}
%
% \usepackage{pygmentize}
% \usepackage{minted}
% \usemintedstyle[python]{friendly}
\usepackage{import} 
%
%\pgfpagesuselayout{4 on 1}[letterpaper,landscape]
\usepackage[font=scriptsize,labelfont=bf,justification=centerlast,format=hang,tableposition=top,skip=0.5pt]{caption} %To change the appearance of captions
\captionsetup[table]{name=Table}
% tikz
\usepackage{tikz} % for draw
\usetikzlibrary{%
  shapes.geometric,
  automata,
  arrows,
  positioning,
  calc,
  intersections}
\graphicspath{{figs/}}
% flowchart with tikz library
\tikzstyle{startstop}=[rectangle, rounded corners, minimum width=3cm, minimum height=1cm, text centered, draw=black, fill=red!30]
%
\tikzstyle{io}=[%
trapezium, trapezium left angle=70, trapezium right angle=110,%
minimum width=3cm, minimum height=1cm,%
text centered, draw=black, fill=blue!30%
]
%
\tikzstyle{process}=[%
rectangle, minimum width=3cm, minimum height=1cm,%
text centered,text width=8cm, draw=black, fill=orange!30%
]
%
\tikzstyle{decision}=[%
diamond, minimum width=2cm, minimum height=2cm,%
text badly centered, text width=5em, draw=black, fill=green!30%
]
%
\tikzstyle{arrow}=[thick, ->, >=stealth]
%%%%%%%%%%%%%%%%%%%%
\definecolor{tealsection}{RGB}{50,90,90}
\hypersetup{colorlinks=true,
  urlcolor=red,
  linkcolor=tealsection}

\usepackage[T1]{fontenc}
\usepackage{listings}
\usepackage{subfiles}
\definecolor{keywords}{RGB}{255,0,90}
\definecolor{comments}{RGB}{60,179,113}
\lstset{
  %inputpath=contents,
  language=python,
  % basicstyle=\scriptsize\fontfamily{fvm}\selectfont,
  basicstyle=\tiny\fontfamily{fvm}\selectfont,
  %basicstyle=\footnotesize\ttfamily,
  keywordstyle=\color{keywords},
  commentstyle=\color{comments}\emph,
  stringstyle=\color{orange},
  upquote=false,
  showstringspaces=false
}
%%% Ipython Notebook Style
%\input{ipythonnb_style}
% ========================================
\newenvironment<>{exercise}[1]
{
\setbeamercolor{block title}{fg=white,bg=orange!45!black}
\begin{block}#2{Ejercicio #1}\justifying
}
{%
\end{block}
}%
% ========================================
\newenvironment<>{solution}[1]
{
\setbeamercolor{block title}{fg=white,bg=orange!65!black}
\begin{block}#2{Ejercicio #1 (Solución)}\justifying
}
{
\end{block}
}
% ========================================
\newenvironment<>{example}[1]
{
\setbeamercolor{block title}{fg=white,bg=green!65!black}
\begin{block}#2{Ejemplo #1}\justifying
}
% content
{
\end{block}
}
% ========================================
\newcounter{examplecounter}
%\newcommand{\refexample}[1]{\refstepcounter{examplecounter}\label{#1}}
\resetcounteronoverlays{examplecounter}
\newenvironment<>{frameExample}[2]
{
\setbeamercolor{frametitle}{fg=white,bg=green!60!blue}
\begin{frame} \justifying
  \frametitle{\refstepcounter{examplecounter}Example \theexamplecounter. #1}
  \framesubtitle{\insertsubsectionhead #2}  
}
% content
{
\end{frame}
}
% ========================================
\newcounter{coding}
\resetcounteronoverlays{coding}
\newenvironment<>{framecode}[1][] % for coding
{
\setbeamercolor{frametitle}{fg=white,bg=black!85}
\begin{frame}[environment=fr,#1]
\frametitle{\refstepcounter{coding} Python Coding Example \thecoding}
\framesubtitle{}
}%
{
\end{frame}
}
% ========================================

\newcounter{activitycount}
\newenvironment<>{frameact}[1] % For Activities
{
  % \usebackgroundtemplate{\includegraphics[scale=0.3]{blackboard-logo.png}}
\setbeamercolor{frametitle}{fg=white,bg=red!60!black}
\begin{frame}\justifying
  \frametitle{\refstepcounter{activitycount} Activity \theactivitycount. #1}
  \framesubtitle{}

}
{
\end{frame}
}
% ========================================

\usetheme[%
sectionpage=progressbar,
numbering=none,
progressbar=frametitle%
]{metropolis}
%\usetheme{Boadilla}

\title{OPERATIONS RESEARCH.}
%\subtitle{Operations Research. \\ Mathematical Models.} 
\subtitle{Operations Research} 

% \institute{UNIVERSIDAD ANÁHUAC MÉXICO}
\author{Rafael Torres Escobar, Ph.D.}
\date[OR]{}





%%%%%%%%%%%%%%%%%%%%%%%%%%%%%%

\begin{document}

\begin{frame}
  \maketitle
\end{frame}


\begin{frame}{Agenda}
  \tableofcontents
\end{frame}
 % ==============================
 % WRITE CONTENT  BELOW
 % %%%

%\begin{frameExample}{10. Fluid Blending Problem}{}
  An oil cmpany produces tow grades of gasoline P and Q which it sells at \$ 30, and \$ 40 per litre. The company can buy four different crude oils withthe following constituents and costs:

  {\centering
    \begin{tabular}{ccccc}
      \toprule
      Crude&\multicolumn{3}{c}{Constituents}& Price / litre\\
      \cmidrule{2-4}
      oil&$A$&$B$&$C$& \$ \\
      \midrule
      1&0.75&0.15&0.10 & 20.00\\
      2&0.20&0.30&0.50&22.50\\
      3&0.70&0.10&0.20&25.00\\
      4&0.40&0.10&0.50&27.50\\
      \bottomrule
    \end{tabular}
    \par}

  Gasoline $P$ must have at least 55 per cent of constituent $A$ and no more than 40 per cent of $C$. Gasoline $Q$ must not have more than 25 per cent of $C$. Determine how the crudes should be used to maximize the profit.
\end{frameExample}


%%% Local Variables:
%%% mode: latex
%%% TeX-master: "../slides_linear-programming-intro"
%%% End:

%\input{contents/graphic-method.tex}

% \section{Método Simplex}
% \label{sec:simplex-method}

% \subfile{03_simplex/slides_simplex}
% %
\section{Forma Canónica y Estándar de un Problema P.L.}
\label{sec:canonical-standard-form}

\begin{frame}{Forma Canónica y Estándar}
  \begin{columns}
    \column{0.5\textwidth}
      \begin{block}{Forma Canónica}
  Maximizar \[ Z = \sum_{j=1}^{n} c_j x_j\] 
  sujeto a
  \begin{align*}
    \sum a_{ij}x_j  & \leq b_i, \quad i = 1, 2, \ldots, m,\\
    x_j  & \geq 0, \quad j = 1, 2, \ldots, n,\\
  \end{align*}  
\end{block}
\column{0.5\textwidth}
\begin{block}{Forma Estándar}
    Maximizar \[ Z = \sum_{j=1}^{n} c_j x_j\] 
  sujeto a
  \begin{align*}
    \sum a_{ij}x_j  & = b_i, \quad i = 1, 2, \ldots, m,\\
    x_j  & \geq 0, \quad j = 1, 2, \ldots, n,\\
  \end{align*}
\end{block}
  \end{columns}
\end{frame}


\begin{frame}{Forma Estándar Para P.P.L.}
  \begin{onlyenv}<1>
    
    Para resolver un Problema de Programación Lineal (P.P.L.), éste se debe expresar en la forma estándar.\\

    Maximizar \[ Z = \sum_{j=1}^{n} c_j x_j\] 
  sujeto a
  \begin{align*}
    \sum a_{ij}x_j  & \leq  b_i, \, (\geq b_i), & i = 1, 2, \ldots, m,\\
    x_j  & \geq 0, & j = 1, 2, \ldots, n,\\
  \end{align*}
\end{onlyenv}
\begin{onlyenv}<2>
  Expresado en forma estándar obtnemos\\

  Maximizar \[ Z = \sum_{j=1}^{n} c_j x_j\] 
  sujeto a
  \begin{align*}
    \sum a_{ij}x_j + s_i & =  b_i,  \,\, i = 1, 2, \ldots, m,\\
    x_j  & \geq 0, \, \, j = 1, 2, \ldots, n,\\
    s_i  & \geq 0, \, \, i = 1, 2, \ldots, m.\\
  \end{align*}
\end{onlyenv}
\end{frame}


\begin{frame}[t]{Ejemplos}
  \begin{columns}[t]
    \column{0.5\textwidth}
    \begin{flalign*}
    \max Z = 7x_1 + 5x_2&\\
    \intertext{sujeto a}
    2x_1 + 3x_2 & \leq 20\\
    3x_1 + x_2 &\geq 10\\
    x_1, x_2 & \geq 0
  \end{flalign*}
  \column{0.5\textwidth}
  \begin{flalign*}
    % EXAMPLE  2.12-2
    \max Z = 3x_1 + 2x_2 + 5x_3&\\
    \intertext{sujeto a}
    2x_1 - 3x_2 & \leq 3\\
    x_1 + 2x_2 + 3x_3 &\geq 5\\
    3x_1 + 2x_3 &\leq 2\\
    x_1, x_2 & \geq 0
  \end{flalign*}
  \end{columns}
\end{frame}

\begin{frameExample}{Variables Irrestrictas}{}
  \begin{columns}[t]
    \column{0.5\textwidth}
    \begin{flalign*}
      % Example 2.12-3 
    \max Z = 3x_1 + 2x_2 + 5x_3&\\
    \intertext{sujeto a}
    2x_1 + 3x_2 - 2x_3& \leq 40\\
    4x_1 - 2x_2 + x_3&\leq 24\\
    x_1 - 5x_2 - 6x_3&\geq 2\\
    x_1 & \geq 0
  \end{flalign*}
  \column{0.5\textwidth}
  \begin{flalign*}
    % EXAMPLE  2.12-4
    \max Z = 2x_1 + 3x_2 &\\
    \intertext{sujeto a}
    2x_1 - 3x_2 - x_3& = -4\\
    3x_1 + 4x_2 - 3x_4 &= -6\\
    2x_1 + 5x_2 + x_5 &= 10\\
    4x_1 - 3x_2 + x_6 &= 18\\
    x_3, x_4,x_5, x_6 & \geq 0
  \end{flalign*}
  \end{columns}
\end{frameExample}



%%% Local Variables:
%%% mode: latex
%%% TeX-master: "../slides"
%%% End:




%%% Local Variables:
%%% mode: latex
%%% TeX-master: "slides_simplex"
%%% End:

\begin{frameact}{Forma Estándar}{}
      \begin{flalign*}
      % Example 2.12-5
    \max Z = 3x_1 + 5x_2 - 2x_3&\\
    \intertext{sujeto a}
    x_1 + 2x_2 - x_3& \geq -4\\
    -5x_1 + 6x_2+ 7x_3 & \geq 5\\
    2x_1 + x_2 + 3x_3&= 10\\
    x_1, x_2 & \geq 0
  \end{flalign*}
\end{frameact}


%%% Local Variables:
%%% mode: latex
%%% TeX-master: "../slides"
%%% End:


%%% Local Variables:
%%% mode: latex
%%% TeX-master: "slides"
%%% End:

%\section{Analytical Method}
\label{sec:simplex-method}




\begin{frame}{Important Definitions}
  \begin{onlyenv}<1>
  Considerar el siguiente problema de programación lineal con $n$ variables y $m$ restricciones:
  \begin{flalign*}
  \max Z = c_1x_1 + c_2x_2 + \cdots + c_nx_n &\\
  \intertext{Subject to}
  a_{11}x_1 + a_{12}x_2 + \cdots + a_{1n}x_n  &\leq b_1,\\
  a_{21}x_1 + a_{22}x_2 + \cdots + a_{2n}x_n  & \leq b_2\\
  \vdots \qquad \qquad\qquad \qquad &\vdots\\
  a_{m1}x_1 + a_{m2}x_2 + \cdots + a_{mn}x_n & \leq b_m
\end{flalign*}
donde $x_1, x_2, , \ldots, x_n \geq 0$

¿Cómo obtenemos la forma estándar? \\ Usar variables de holgura $x_{n+1}, x_{n+2}, \ldots, x_{n+m}$
\end{onlyenv}
\end{frame}

\begin{frame}{Theory of Simplex Method}{}

      \begin{columns}[t]
    \column{0.3\textwidth}    
    \begin{block}{Condiciones fundamentales}
  \begin{enumerate}  \justifying \parskip3mm
  \item Factibilidad
  \item Optimalidad
  \end{enumerate}
\end{block}

\column{0.7\textwidth}
\begin{block}{Consideraciones Importantes}
  \begin{enumerate} \justifying 
  \item Solución.
  \item Solución Factible.
  \item Solución Básica.
  \item Solución Básica Factible.
  \item Solución Básica Factible No Degenerada.
  \item Solución Básica Factible Degenerada.
  \item Solución Básica Factible Óptima.
  \item Solución No Acotada.
  \item Conjunto de Puntos
  \item Conjutno Convexo
  \end{enumerate}
\end{block}
\end{columns}
\end{frame}

% ---
\begin{frame}{Teoría Del Método Simplex}
  \begin{columns}
    \column{0.3\textwidth}
\begin{onlyenv}<1,2>
    \begin{flalign*}
    \max Z = 3x_1 + 4x_2&\\
    \intertext{Subject to}
    x_1 + x_2 &\leq 450\\
    2x_1 + x_2& \leq 600\\
    x_1, x_2  &\geq 0
  \end{flalign*}
\end{onlyenv}
    \column{0.7\textwidth}
      {\centering
\includegraphics<1>[scale=0.4]{fig_example-simplex01.pdf}
\par}


\begin{onlyenv}<2>
  El número de soluciones básicas con $m$ ecuaciones y $m + n$ variables es:\[ _{m +n}C_{m} = \frac{(m + n)!}{m!n!} \]
\end{onlyenv}
  \end{columns}
\end{frame}

% ---------
\begin{frame}{Trial and Error Approach}
  \begin{columns}[t]
    \column{0.4\textwidth}
    \begin{onlyenv}<1>
          \begin{flalign*}
    \max Z = 3x_1 + 4x_2&\\
    \intertext{Subject to}
    x_1 + x_2 &\leq 450\\
    2x_1 + x_2& \leq 600\\
    x_1, x_2  &\geq 0
  \end{flalign*}
\end{onlyenv}

\begin{onlyenv}<2>
  \begin{align*}
    _{m +n}C_{m} &= \frac{(m + n)!}{m!n!}  \\[3mm]
    _{4}C_{2} &= \frac{4!}{2!2!}  \\[2mm]
    _{4}C_{2} &  = 6
  \end{align*}
\end{onlyenv}
    \column{0.4\textwidth}
    \begin{align*}
    \max Z = 3x_1 &+& 4x_2 &+& 0s_1 &+& 0s_2 & &\\
    \intertext{Subject to}
    x_1 &+& x_2 &+& s_1 && &= & 450\\
    2x_1 &+& x_2 & & &+&  s_2 &=& 600\\
    x_1&,& x_2&,& s_1&, &s_2  &\geq& 0
  \end{align*}
\end{columns}
\end{frame}

  %% ------------------------------
\begin{frame}{Método Prueba y Error}{}
            \begin{align*}
    \max Z = 3x_1 &+& 4x_2 &+& 0s_1 &+& 0s_2 & &\\
    \intertext{Subject to}
              \textcolor<1,2,3>{red}{x_1} &+& \textcolor<1,4,5>{red}{x_2} &+& \textcolor<2,4,6>{red}{1s_1} &+&\textcolor<3,5,6>{red}{\cancel{0s_2}} &= & \textcolor{red}{450}\\
          \textcolor<1,2,3>{red}{2x_1} &+& \textcolor<1,4,5>{red}{x_2} &+ &\textcolor<2,4,6>{red}{\cancel{0s_1}} &+&  \textcolor<3,5,6>{red}{1s_2} &=& \textcolor{red}{600}\\
    %x_1&,& x_2&,& s_1&, &s_2  &\geq& 0
            \end{align*}
            \begin{columns}[t]
              \column{0.4\textwidth}
              \begin{align*}
              %\max Z = 3x_1 + 4x_2 + 0s_1 + 0s_2&\\
              %\intertext{Subject to}
              \only<1,2,3>{1 \textcolor{red}{x_1} }  \only<1,2,3>{+} \only<1,4,5>{1 \textcolor{red}{x_2}} \only<4,5>{+} \only<2,4,6>{1 \textcolor{red}{s_1}} \only<6>{+}  \only<3,5,6>{0 \textcolor{red}{s_2}}&= 450\\
              \only<1,2,3>{2 \textcolor{red}{x_1} } \only<1,2,3>{+}  \only<1,4,5>{1 \textcolor{red}{x_2}} \only<4,5>{+} \only<2,4,6>{0 \textcolor{red}{s_1}} \only<6>{+} \only<3,5,6>{1 \textcolor{red}{s_2}}& = 600\\
              %x_1, x_2  &\geq 0
              \end{align*}
              \column{0.2\textwidth}
              \begin{align*}
                Z &= \only<1>{1650} \only<2>{900} \only<3>{\textcolor{cyan}{--}} \only<4>{\textcolor{cyan}{--}} \only<5>{\textcolor{red}{1800}} \only<6>{0}\\
                \textcolor<1,2,3>{red}{x_1} &= \textcolor{red}{\only<1>{150} \only<2>{300} \only<3>{450}}  \only<4,5,6>{0}\\
                \textcolor<1,4,5>{red}{x_2} &= \textcolor{red}{\only<1>{300} \only<4>{600} \only<5>{450}}  \only<2,3,6>{0}\\
              \end{align*}
              \column{0.2\textwidth}
              \begin{align*}
                \\
                \textcolor<2,4,6>{red}{s_1} &=  \only<4>{\textcolor{cyan}{-150}} \textcolor{red}{\only<2>{150}  \only<6>{450}}\only<1,3,5>{0}\\
                \textcolor<3,5,6>{red}{s_2} &= \only<3>{\textcolor{cyan}{-300}}  \textcolor{red}{ \only<5>{150} \only<6>{600}} \only<1,2,4>{0}
              \end{align*}
            \end{columns}
  \end{frame}
  %% ------------------------------

\begin{frameExample}{Método de Prueba y Error}{}
  % EXAMPLE  2.15-2 Gupta Ebook
  \begin{onlyenv}<1>
    \begin{flalign*}
    \max Z = x_1 + 3x_2 + 3x_3&\\
    \intertext{Subject to}
    x_1 + 2x_2 + 3x_3& = 4\\
    2x_1 + 3x_2 + 5x_3& = 7\\[3mm]
    x_1, x_2 & \geq 0\\
    x_3 & \text{  irrestricta en signo}
  \end{flalign*}
\end{onlyenv}

\begin{exampleblock}<only@2>{Planteamiento} \justifying

  Existe una variable irrestricta $x_3$, por lo tanto hacemos  $ x_3 = x_4  - x_5 $
  \begin{flalign*}
    \max Z = x_1 + 3x_2 + 3x_4  - 3x_5&\\
    \intertext{Subject to}
    x_1 + 2x_2 + 3x_4 - 3x_5& = 4\\
    2x_1 + 3x_2 + 5x_4 - 5x_5& = 7\\
    x_1, x_2, x_4, x_5 & \geq 0
  \end{flalign*}
\end{exampleblock}
\end{frameExample}

\begin{frameExample}{Prueba y Error. Minimización}{}
  \begin{columns}
    \column{0.5\textwidth}
  % EXAMPLE  2.09-7 Gupta Ebook
   \begin{align*}
     \min Z = -x_1 + 2x_2 & \\[5mm]
     -x_1 + 3x_2 & \leq 10\\
     x_1 + x_2 & \leq 6\\
     x_1 - x_2 & \leq 2\\[5mm]
     x_1, x_2 & \geq 0
  \end{align*}
  \column{0.5\textwidth}
  \begin{align*}
     \min Z = -x_1 + 2x_2 + 0s_1 + 0s_2 + 0s_3 & \\[5mm]
     -x_1 + 3x_2 + s_1 & = 10\\
     x_1 + x_2 + s_2& = 6\\
     x_1 - x_2 + s_3& = 2\\[5mm]
     x_1, x_2, s_1, s_2, s_3 & \geq 0
  \end{align*}
  \end{columns}
\end{frameExample}

%%% Local Variables:
%%% mode: latex
%%% TeX-master: "slides_simplex"
%%% End:

\begin{frameExample}{Método de Prueba y Error}{}
  % EXAMPLE  2.15-3 Gupta Ebook
  Una empresa fabrica cuatro partes diferentes de una máquina $M_1, M_2,$ $ M_3, M_4$ con una aleación de cobre y zinc. Las cantidades de cobre y zinc requeridas para cada parte, su \alert{disponibilidad exacta} y la ganancia de cada unidad para cada parte es la siguiente

  {\centering
    \begin{tabular}{rccccc}
      \toprule
      &&&&&Disponibilidad\\
      &$M_1$&$M_2$&$M_3$&$M_4$&exacta\\
      &(kg)&(kg)&(kg)&(kg)&(kg)\\
      \midrule
      Cobre&5&4&2&1&100\\
      Zinc&2&3&8&1&75\\
      Ganancia&12&8&14&10&\\
      \bottomrule
    \end{tabular}
  \par}

¿Cuántas partes se deben fabricar para maximizar la ganancia?
\end{frameExample}

%%% Local Variables:
%%% mode: latex
%%% TeX-master: "slides_simplex"
%%% End:

\begin{frameExample}{Método de Prueba y Error}{}
  % EXAMPLE  2.15-04 Gupta Ebook
  En el siguiente sistema de ecuaciones lineales existen soluciones degeneradas. Usar matrices para ver cuáles son las soluciones degeneradas.

  \begin{columns}
    \column{0.3\textwidth}
      \begin{flalign*}
    2x_1 + x_2 - x_3 &= 2\\
    3x_1 + 2x_2 + x_3 &=3
  \end{flalign*}
  \column{0.7\textwidth}
  \[\bm{A} =%
    \begin{bmatrix}
      2 & 1 & -1\\
      3 & 2 & 1 \\
    \end{bmatrix},
    \bm{x} = %
    \begin{bmatrix}
      x_1\\
      x_2\\
      x_3
    \end{bmatrix},
    \bm{b} = %
    \begin{bmatrix}
      2\\
      3
    \end{bmatrix}
  \]
  \end{columns}
\end{frameExample}

%%% Local Variables:
%%% mode: latex
%%% TeX-master: "../slides"
%%% End:

\begin{frameExample}{Trial and Error Method}{}

\[    \max Z = 2x_1 + 3x_2  \]
{
  \centering
  subject to

  \sysdelim..%
  \sysalign{r,r}%
  \systeme[x_1x_2]%
  {
    x_1 + x_2  \leq 30,
    x_2  \geq 3,
    x_2  \leq 12,
    x_1 - x_2  \geq 0,
    x_1   \leq 20
  }

  \vspace{5mm}
  $    x_1, x_2  \geq 0$
  \par
}
\end{frameExample}


%%% Local Variables:
%%% mode: latex
%%% TeX-master: "slides_simplex"
%%% End:

\begin{frameExample}{Trial And Error Method  \label{example:2-9-3_Gupta-ebook}}{}

  % EXAMPLE  2.9-3 Gupta ebook
  \[     \max Z = 2x_1 + x_2 \]

  {\centering
    subject to

    \sysdelim..%
    \sysalign{r,r}%
    \systeme[x_1x_2]%
    {
      x_1 + 2x_2  \leq 10,
      x_1 + x_2  \leq 6,
      x_1 - x_2  \leq 2,
      x_1 - 2x_2  \leq 1
    }

    \vspace{5mm}
    $    x_1, x_2  \geq 0$
    \par}


\end{frameExample}

%%% Local Variables:
%%% mode: latex
%%% TeX-master: "slides"
%%% End:

\begin{frameExample}{Producción}{}
  % EXAMPLE 2.6-1 (Production Allocation Problem} Gupta ebook
  Una empresa produce tres productos. Estos productos se procesan en tres máquinas diferentes. El tiempo requerido para fabricar una unidad de cada uno de los tres productos y la capacidad diaria de las tres máquinas se detallan en la tabla a continuación.

  {\centering
    \scalebox{0.8}{%
      \begin{tabular}{ccccc}
        \toprule
        Máquina & \multicolumn{3}{l}{Tiempo por unidad (minutos)} & Capacidad       \\
                &  Producto 1             &    Producto 2            &     Producto 3           & (Minutos / día) \\
        \midrule
        $M_1$   & 2             & 3              & 2              & 440             \\
        $M_2$   & 4             & --             & 3              & 470             \\
        $M_3$   & 2             & 5              & --             & 430\\
        \bottomrule
      \end{tabular}
    }% scalebox
    \par}
  
  

   Se requiere determinar la cantidad diaria de unidades que se fabricarán para cada producto. El beneficio por unidad para el producto 1, 2 y 3 es de \$ 4, \$ 3 y \$ 6 respectivamente. Se supone que todas las cantidades producidas se consumen en el mercado. Formule el modelo matemático (L.P.) que maximizará la ganancia diaria.
\end{frameExample}



%%% Local Variables:
%%% mode: latex
%%% TeX-master: "slides_simplex"
%%% End:

\begin{frameExample}{02. Dieta}{}
  % EXAMPLE 2.6-2 {Diet Problem} Gupta
  La persona quiere decidir los componentes de una dieta que cumpla con sus requerimientos diarios de proteínas, grasas y carbohidratos al costo mínimo. La elección debe hacerse a partir de cuatro tipos diferentes de alimentos. Los rendimientos por unidad de estos alimentos se dan en la tabla siguiente. Formule un modelo de programación lineal para el problema.
  
  {\centering
    \scalebox{0.8}{
      \begin{tabular}{ccccc}
        \toprule
        Food&\multicolumn{3}{c}{Yield per unit}&Cost per\\
        \cmidrule{2-4}
        Type&Proteins&Fats&Carbohydrates&unit (\$)\\
        \midrule
1&3&2&6&45\\
2&4&2&4&40\\
3&8&7&7&85\\
        4&6&5&4&65\\
        \bottomrule
Minimum&&&&\\
requirement&800&200&700&\\
\bottomrule
      \end{tabular}
}
\par}

  
\end{frameExample}



%%% Local Variables:
%%% mode: latex
%%% TeX-master: "../slides"
%%% End:




%%% Local Variables:
%%% mode: latex
%%% TeX-master: "slides_simplex"
%%% End:


% ==============================
\begin{frame}
  \maketitle
\end{frame}


\end{document}
