\begin{frameExample}{Método de Prueba y Error}{}
  % EXAMPLE  2.15-3 Gupta Ebook
  Una empresa fabrica cuatro partes diferentes de una máquina $M_1, M_2,$ $ M_3, M_4$ con una aleación de cobre y zinc. Las cantidades de cobre y zinc requeridas para cada parte, su \alert{disponibilidad exacta} y la ganancia de cada unidad para cada parte es la siguiente

  {\centering
    \begin{tabular}{rccccc}
      \toprule
      &&&&&Disponibilidad\\
      &$M_1$&$M_2$&$M_3$&$M_4$&exacta\\
      &(kg)&(kg)&(kg)&(kg)&(kg)\\
      \midrule
      Cobre&5&4&2&1&100\\
      Zinc&2&3&8&1&75\\
      Ganancia&12&8&14&10&\\
      \bottomrule
    \end{tabular}
  \par}

¿Cuántas partes se deben fabricar para maximizar la ganancia?
\end{frameExample}

%%% Local Variables:
%%% mode: latex
%%% TeX-master: "../slides"
%%% End:
