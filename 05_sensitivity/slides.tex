\documentclass[../main.tex]{subfiles}
\graphicspath{{figs/}}
\title{Sensitivity Analysis}

\AtBeginSection[] % Do nothing for \section* %
{
\begin{frame}<beamer> 
  \frametitle{Agenda}
  \tableofcontents[currentsection] 
\end{frame}
}


\begin{document}
% ==============================
\begin{frame}
  \maketitle
\end{frame}


\begin{frame}{Agenda}
  \tableofcontents
\end{frame}
% ==============================

\section{What Is Sensitivity Analysis in L.P.?}
\label{sec:sensitivityintro}


\begin{frame}{Sensitivity Analysis}
  \begin{itemize} \justifying \parskip4mm
    \item<only@1> It is desirable to study how the current optimal solution changes when the parameters of the problem get changed. In these problems this inoformation may be more important than the single result provided by the optimal solution.
    \item<only@1> After attaining the optimal solution, one may discover that \alert{a wrong value of a cost coefficient was used or a particular variable or constratint was ommited} or one or more of tight-hand constants used were wrong.
    \item<only@1> The changes in parameters of the problem may be \alert{discrete or continuous}. The study of the effect of \alert{discrete changes in parameters on the optimal solution} is called the \alert{sensitivity analysis} or \alert{post optimality analysis}, while continuous changes in parameters is called \alert{parametric programming}.
    \item<only@2> The changes in the parameters of a linear programming problem include:%
          \begin{enumerate} \justifying
            \item Changes in the right-hand side of the constraints or availability of resources ($b_i$)
            \item Changes in the cost/profit coefficients or cost / profit contribution per unit of decision variables ($c_j$).
            \item Addition of new variables
            \item Changes in the coefficients of constraints or consumption of resources per unit of decision variables $a_{ij}$.
            \item Addition of new constraints.
            \item Deletion of variables.
            \item Deletion of constraints.
          \end{enumerate}
  \end{itemize}
\end{frame}


\section{Changes in The Right-Hand Side}
\label{sec:changes-right-hand-side}

\begin{frame}{Changes in The Right-Hand Side of The Constraints $b_i$}{}
  \begin{itemize} \justifying \parskip3mm
  \item Suppose that an optimal solution to a linear programming has already been found and it is desired to find the effect of increasing or decreasing some resource.
  \item This will affect not only the objective function but also the solution.
  \item Large changes in the limiting resources \alert{may even change the variables in the solution since one or more current basic variables become negative}.
  \item Dual simplex method is used to remove infeasibility and to get a feasbile optimal solution.
  \end{itemize}
  
\end{frame}

\begin{frameExample}{}{}

  \begin{columns}
    \column{0.4\textwidth}
      $\max Z = 5x_1 + 12x_2 + 4x_3$

      \vspace{5mm}

      s.t.
  \sysalign{r,r}%
  \systeme[x_1x_2x_3]{%
    x_1 + 2x_2 + x_3 \leq 5,
    2x_1 - x_2 +3x_3 = 2  
  }

  \vspace{5mm}
  
  $x_1 , x_2, x_3 \geq 0$
  \column{0.6\textwidth}
  \begin{enumerate}[a)] \justifying \parskip4mm
  \item Solve the problem.
  \item Discuss the effect of changing the requirement vector from $%
    \begin{bmatrix}
      5\\
      2\\
    \end{bmatrix}
    $%
    to%
    $
    \begin{bmatrix}
      7\\
      2\\
    \end{bmatrix}
$
\item Discuss the effect of changing the requirement vector from %
  $
  \begin{bmatrix}
    5\\2\\
  \end{bmatrix}
  $
  to%
  $
  \begin{bmatrix}
    3\\9\\
  \end{bmatrix}
  $
  \item Which resource should be increased and how much to achieve the best marginal increase in the value of the objective function?
  \end{enumerate}
  \end{columns}
\end{frameExample}


\begin{frameExample}{}{}
\begin{columns}
  \column{0.5\textwidth}
  Given the following L.P. Problem:

  \[ \max Z = -x_1 + 2x_2 - x_3 \]

  \vspace{6mm}
  s.t.
  \sysalign{r,r}%
  \systeme[x_1x_2x_3]%
  {
    3x_1 + x_2 - x_3 \leq 10,
    -x_1 + 4x_2 + x_3 \geq 6,
    x_2 + x_3 \leq 4
  }

  \vspace{6mm}

  $x_1, x_2, x_3 \geq 0$
  \column{0.5\textwidth}
  \begin{enumerate}[a)] \justifying \parskip3mm
  \item Find the optimal solution.
  \item   Find the separate ranges $b_1, b_2 $ and $b_3$ consistent with the optimal solution.
  \end{enumerate}
\end{columns}  
\end{frameExample}


%%% Local Variables:
%%% mode: latex
%%% TeX-master: "slides_sensitivity"
%%% End:


\section{Changes in the Cost/Profit Coefficients $c_j$}
\label{sec:changes-cost-profit}

\begin{frame}{Changes in the Cost/Profit Coefficients $c_j$}
  \begin{itemize}\justifying \parskip3mm
    \item Changes in the coefficiets of the objective function may take place due to a change in cost or profit of either \alert{basic variables or non-basic variables}.
    \item Each of these two cases will first be considered separately. The discussion, be followed by a combined case.
  \end{itemize}
\end{frame}

\begin{frameExample}{\label{example:6.6-2.1}}{}
  % Example 6.6-2.1
  \begin{onlyenv}<1>
      A company wants to produce three products: $A, B, C$. The unit profits on these products are \$ 4, 6, and 2 respectively. These products require two types of resources -- manpower and material. The following L.P. model is formulated for determining the optimal product mix

  \[\max Z = 4x_1 +6x_2 +2x_3 \]

  {\centering
    subject to

    \vspace{3mm}
    \sysalign{r,r}%
    \sysdelim..%
  \systeme[x_1x_2x_3]%
  {
    x_1 + x_2 + x_3 \leq 3@(manpower),
    x_1 +4x_2 +7x_3 \leq 9@ (material)
  }

  $x_1, x_2, x_3 \geq 0$
  \par}

  Where   $x_1, x_2, x_3$ are the number of products $A, B $ and $C$ produced.
  \end{onlyenv}

\begin{onlyenv}<2>
  \begin{enumerate}[a)] \justifying \parskip3mm
  \item Find the optimal product mix and the corresponding profit to the company.
  \item Find the range on the values of non-basic variable coefficient $c_3$ such that the current optimal product mix remains optimal.
  \item What happens if $c_3$ is increased to \$ 12? What is the new optimal product mix in this case?
  \item Find the range on basic variable coefficient $c_1$ such that the current optimal product mix remains optimal.
  \item Find the effect when $c_1 = \$8$ on the optimal product mix.
  \item Find the effect of changing the objective function to $Z = 2x_1 + 8x_2 + 4x_3$ on the current optimal product mix.
  \end{enumerate}
\end{onlyenv}
\end{frameExample}



\begin{frameExample}{}{}

      \begin{columns}
    \column{0.5\textwidth}
    Given the L.P. Problem:

    $\max Z = -x_1 + 2x_2 - x_3$

    
    s.t.

    \vspace{5mm}
    \sysalign{r,r}%
    \sysdelim..%
    \systeme[x_1x_2x_3]%
    {
      3x_1 + x_2 - x_3 \leq 10,
      -x_1 + 4x_2 + x_3 \geq 6,
      x_2 + x_3 \leq 4 
    }

    \vspace{5mm}

    $x_1,x_2,x_3 \geq 0$
    \column{0.5\textwidth}

    determine the effect of discrete changes in $c_j (j = 1,2,\ldots,6)$ on the optimal basic feasible solution.
  \end{columns}

\end{frameExample}

\begin{frameExample}{}{}

  \begin{columns}[t]
    \column{0.5\textwidth}
    Consider the L.P. Problem:

    \[ \max Z = 3x_1 + 5x_2 + 4x_3\]
    
    \begin{equation*}
          \text{s.t.}
      \sysalign{r,r}%
    %\sysdelim..%
    \systeme[x_1x_2x_3]%
    {
      2x_1 + 3x_2  \leq 8,
      2x_2 + 5x_3 \leq 10,
      3x_1 + 2x_2 + 4x_3 \leq 15 
    }
    \end{equation*}

    \vspace{5mm}

    $x_1,x_2,x_3 \geq 0$
    \column{0.5\textwidth}
    \begin{enumerate}[a)] \parskip3mm \justifying
    \item How much $c_3$ and $c_4$ can be increased till the optimal solution ceases to be optimal? Also find the new value of the objective function if possible.
    \item Find the range over which $b_2$ can be changed maintaining the feasibility of the solution.
    \end{enumerate}
  \end{columns}
\end{frameExample}

%%% Local Variables:
%%% mode: latex
%%% TeX-master: "slides_sensitivity"
%%% End:


\section{Addition of a New Variable.}
\label{sec:addition-new-variable}


\begin{frame}{Addition of a New Variable}{}
  Addition of a new variable in physical sense means introduction of a new product tothe current product mix. Intuitively, it is desirable only if it is profitable \emph{i.e.,} if it improves the optimal value of the objective function.


\end{frame}

  \begin{frameExample}{}{}
    Referring to example~\ref{example:6.6-2.1}, let us suppose the Research and Development department of the company has proposed a fourth product $D$ which requires 1 unit of manpower and 1 unit of material and earns a unit profit of \$ 3 when sold in the market. It is desired to find whether it is profitable to produce product $D$.

    \begin{onlyenv}<1>
            \[\max Z = 4x_1 +6x_2 +2x_3 \]

  {\centering
    subject to

    \vspace{3mm}
    \sysalign{r,r}%
    \sysdelim..%
  \systeme[x_1x_2x_3]%
  {
    x_1 + x_2 + x_3 \leq 3@(manpower),
    x_1 +4x_2 +7x_3 \leq 9@ (material)
  }

  $x_1, x_2, x_3 \geq 0$
  \par}

  Where   $x_1, x_2, x_3$ are the number of products $A, B $ and $C$ produced.
\end{onlyenv}

\begin{onlyenv}<2>
        \[\max Z = 4x_1 +6x_2 +2x_3 + 3x_4 \]

  {\centering
    subject to

    \vspace{3mm}
    \sysalign{r,r}%
    \sysdelim..%
  \systeme[x_1x_2x_3x_4]%
  {
    x_1 + x_2 + x_3 + x_4 \leq 3@(manpower),
    x_1 +4x_2 +7x_3 + x_4\leq 9@ (material)
  }

  $x_1, x_2, x_3, x_4 \geq 0$
  \par}

  Where   $x_1, x_2, x_3, x_4$ are the number of products $A, B, C $ and $D$ produced.
\end{onlyenv}
  \end{frameExample}

  \begin{frameExample}{\label{example:6.6-3.2}}{}
    % Example 6.6-3.2 gupta ebook
    \begin{onlyenv}<1>
      Consider the problem: 

    \[ \max Z = 45x_1 +100x_2 +30x_3 + 50x_4 \]

    subject to
    
    {\centering
  \sysalign{r,r}%
  \sysdelim..%
  \systeme[x_1x_2x_3x_4]%
  {
    7x_1 + 10x_2 + 4x_3 + 9x_4 \leq 1200,
    3x_1 +40x_2 +x_3 + x_4 \leq 800
  }

  \vspace{5mm}
  $x_1, x_2, x_3,x_4 \geq 0$    
  \par}

If the new variable $x_5$ is added to his problem with a column $
\begin{bmatrix}
  10\\10
\end{bmatrix}
$ %
and $c_5 = 120$ find the change in the optimal solution.
\end{onlyenv}
\begin{onlyenv}<2>
   \[ \max Z = 45x_1 +100x_2 +30x_3 + 50x_4 + 120x_5\]

    subject to
    
    {\centering
  \sysalign{r,r}%
  \sysdelim..%
  \systeme[x_1x_2x_3x_4x_5]%
  {
    7x_1 + 10x_2 + 4x_3 + 9x_4 + 10x_5 \leq 1200,
    3x_1 +40x_2 +x_3 + x_4  + 10x_5\leq 800
  }

  \vspace{5mm}
  $x_1, x_2, x_3,x_4, x_5\geq 0$    
  \par}
\end{onlyenv}
  \end{frameExample}


%%% Local Variables:
%%% mode: latex
%%% TeX-master: "slides"
%%% End:

\section{Changes in the Coefficients of the Constraints (Technological Coefficients) $a_{ij}$}

\begin{frame}{Changes in the Coefficients of the Constraints}{}
  \begin{itemize} \justifying \parskip5mm
  \item When changes take place in the coefficients of a \alert{non-basic variable} in a current optimal solution, feasibility of the solution is not affected. The only effect, if any, may be on the optimality of the solution.
  \item If hte constraint coefficients of a \alert{basic variable} get changed, things become more complicated since the feasibility of the current optimal solution may also be affected (lost). \alert{The basic matrix is affected}, which, in turn, may affect all the quantities given in the current optimal table. Under such circumstances, \alert{it may be better to solve the problem all over again}.
  \end{itemize}
\end{frame}

\begin{frameExample}{}{}
  Find the effect of the following changes in the original optimal table of example~\ref{example:6.6-3.2}:

  \begin{enumerate}[a)] \justifying \parskip4mm
  \item $x_1$ column in the problem chaanges from %
    $
    \begin{bmatrix}
      7\\3\\
    \end{bmatrix}
    $ to %
    $
    \begin{bmatrix}
      7\\5\\
    \end{bmatrix}
    $
  \item $x_1$ column in the problem chaanges from %
    $
    \begin{bmatrix}
      7\\3\\
    \end{bmatrix}
    $ to %
    $
    \begin{bmatrix}
      5\\8\\
    \end{bmatrix}
    $
  \end{enumerate}
\end{frameExample}



%%% Local Variables:
%%% mode: latex
%%% TeX-master: "slides"
%%% End:


\section{Addition of a New Constraint}
\label{sec:addition-new-constraint}

\begin{frame}{Addition of a New Constraint}
  \begin{itemize} \justifying \parskip4mm
  \item Addition of a new constraint may o may not affect the feasibility of the current optimal solution.
  \item It is sufficient to check whether new constraint is satified by the current optimal solution or not.
  \item If it is satisfied, the inclusion of the constraint has no effect on the current optimal solution \emph{i.e.,} it remains feasible as well as optimal. If, however, the constraint is not satified, the current \alert{optimal solution becomes infeasible. Dual simplex metho is then used to find the new optimal solution.}
  \end{itemize}
\end{frame}

\begin{frameExample}{\label{example:6.6-5.1}}{}
  % Example 6.6-5.1 gupta ebook
  \begin{enumerate}[a)] \justifying \parskip4mm
  \item In example~\ref{example:6.6-2.1} an administrative constraint is added. Products $A, B$ and $C$ require 2, 3, and 2 hours of administrative services, while the total available administrative hours are 10. How does the optimal solution change?
  \item \label{example08} If the total available administrative time is 4 hours, find the new optimal solution.
  \end{enumerate}
\end{frameExample}

%%% Local Variables:
%%% mode: latex
%%% TeX-master: "slides_sensitivity"
%%% End:


\section{Deletion Of A Variable}
\label{sec:deletion-variable}

\begin{frame}{Deletion Of A Variable}{}
  \begin{itemize} \justifying \parskip3mm
  \item Deletion of a \alert{non-basic variable} is a totally superflous operation and does not affect the feasibility or optimality of the current optimal solution.
  \item Deletion of a \alert{basic variable} may affect the optimality and a new optimum solution may have to be found out.
  \item A \alert{heavy penalty} $-M$ ($+M$ for minimization problems) is assigned to the variable under consideration and the new optimum solution is obtained by \alert{applying regular simplex method} to the (modified) current optimum table.
  \end{itemize}
\end{frame}

\begin{frameExample}{\label{example:6.6-6.1}}{}
  Consider the optimal solution of example~\ref{example:6.6-5.1} when a new  constratint was added. If product $B$ is not to be produced, so that variable $x_2$ is to be deleted from this table, find the optimum solution to the resulting L.P. problem.
  
\end{frameExample}


%%% Local Variables:
%%% mode: latex
%%% TeX-master: "slides"
%%% End:


\section{Deletion of a Constraint}
\label{sec:deletion-constraint}

\begin{frame}{Deletion of a constraint}{}
  \begin{itemize} \justifying \parskip3mm
\item<only@1> The constraint to be deleted may be either binding or unbinding on the optimal solution. The deletion of an unbinding constraint can only enlarge the feasible region but will no affect the optimal solution.
\item<only@1> If the constraint under consideration has a slack os surplus variable of zero value in the basis matrix, it cannot be binding and hence will not affect the optimal solution.
\item<only@2> The deletion of a binding constraint will, however, cause postoptimality problem. The simplest way to proceed in this case is via the addition of one or two new variables. For example, the constraint \[ a_{11}x_1  + a_{12}x_2 + \ldots + a_{i n}x_n = b_{i}\] can be written as \[ a_{11}x_1  + a_{12}x_2 + \ldots + a_{i n}x_n + x_{n+1} - x_{n+2} = b_{i};\,\, x_{n+1}, x_{n+2} \geq 0\] where $x_{n+1}, x_{n+2}$ are slack and surplus variables respectively. The problem can then be solvled by the procedure laid down in the section \hyperlink{sec:addition-new-variable}{addition of new variables}.
\end{itemize}
\end{frame}


%%% Local Variables:
%%% mode: latex
%%% TeX-master: "slides"
%%% End:

\end{document}
