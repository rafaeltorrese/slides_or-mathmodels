
\section{Addition of a New Variable.}
\label{sec:addition-new-variable}


\begin{frame}{Addition of a New Variable}{}
  Addition of a new variable in physical sense means introduction of a new product tothe current product mix. Intuitively, it is desirable only if it is profitable \emph{i.e.,} if it improves the optimal value of the objective function.


\end{frame}

  \begin{frameExample}{}{}
    Referring to example~\ref{example:6.6-2.1}, let us suppose the Research and Development department of the company has proposed a fourth product $D$ which requires 1 unit of manpower and 1 unit of material and earns a unit profit of \$ 3 when sold in the market. It is desired to find whether it is profitable to produce product $D$.

    \begin{onlyenv}<1>
            \[\max Z = 4x_1 +6x_2 +2x_3 \]

  {\centering
    subject to

    \vspace{3mm}
    \sysalign{r,r}%
    \sysdelim..%
  \systeme[x_1x_2x_3]%
  {
    x_1 + x_2 + x_3 \leq 3@(manpower),
    x_1 +4x_2 +7x_3 \leq 9@ (material)
  }

  $x_1, x_2, x_3 \geq 0$
  \par}

  Where   $x_1, x_2, x_3$ are the number of products $A, B $ and $C$ produced.
\end{onlyenv}

\begin{onlyenv}<2>
        \[\max Z = 4x_1 +6x_2 +2x_3 + 3x_4 \]

  {\centering
    subject to

    \vspace{3mm}
    \sysalign{r,r}%
    \sysdelim..%
  \systeme[x_1x_2x_3x_4]%
  {
    x_1 + x_2 + x_3 + x_4 \leq 3@(manpower),
    x_1 +4x_2 +7x_3 + x_4\leq 9@ (material)
  }

  $x_1, x_2, x_3, x_4 \geq 0$
  \par}

  Where   $x_1, x_2, x_3, x_4$ are the number of products $A, B, C $ and $D$ produced.
\end{onlyenv}
  \end{frameExample}

  \begin{frameExample}{\label{example:6.6-3.2}}{}
    % Example 6.6-3.2 gupta ebook
    \begin{onlyenv}<1>
      Consider the problem: 

    \[ \max Z = 45x_1 +100x_2 +30x_3 + 50x_4 \]

    subject to
    
    {\centering
  \sysalign{r,r}%
  \sysdelim..%
  \systeme[x_1x_2x_3x_4]%
  {
    7x_1 + 10x_2 + 4x_3 + 9x_4 \leq 1200,
    3x_1 +40x_2 +x_3 + x_4 \leq 800
  }

  \vspace{5mm}
  $x_1, x_2, x_3,x_4 \geq 0$    
  \par}

If the new variable $x_5$ is added to his problem with a column $
\begin{bmatrix}
  10\\10
\end{bmatrix}
$ %
and $c_5 = 120$ find the change in the optimal solution.
\end{onlyenv}
\begin{onlyenv}<2>
   \[ \max Z = 45x_1 +100x_2 +30x_3 + 50x_4 + 120x_5\]

    subject to
    
    {\centering
  \sysalign{r,r}%
  \sysdelim..%
  \systeme[x_1x_2x_3x_4x_5]%
  {
    7x_1 + 10x_2 + 4x_3 + 9x_4 + 10x_5 \leq 1200,
    3x_1 +40x_2 +x_3 + x_4  + 10x_5\leq 800
  }

  \vspace{5mm}
  $x_1, x_2, x_3,x_4, x_5\geq 0$    
  \par}
\end{onlyenv}
  \end{frameExample}


%%% Local Variables:
%%% mode: latex
%%% TeX-master: "slides"
%%% End: