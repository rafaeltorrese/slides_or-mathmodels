
\section{Addition of a New Variable.}
\label{sec:addition-new-variable}


\begin{frame}{Addition of a New Variable}{}
  Addition of a new variable in physical sense means introduction of a new product tothe current product mix. Intuitively, it is desirable only if it is profitable \emph{i.e.,} if it improves the optimal value of the objective function.


\end{frame}

  \begin{frameExample}{05.}{}
    Referring to example 03, let us suppose the Research and Development department of the company has proposed a fourth product D which requires 1 unit of manpower and 1 unit of material and earns a unit profit of \$ 3 when sold in the market. It is desired to find whether it is profitable to produce product $D$.
  \end{frameExample}

  \begin{frameExample}{06.}{}
    Consider the problem:
    

    \[ \max Z = 45x_1 +100x_2 +30x_3 + 50x_4 \]

    subject to
    
    {\centering
  \sysalign{r,r}%
  \sysdelim..%
  \systeme[x_1x_2x_3x_4]%
  {
    7x_1 + 10x_2 + 4x_3 + 9x_4 \leq 1200,
    3x_1 +40x_2 +x_3 + x_4 \leq 800
  }

  \vspace{5mm}
  $x_1, x_2, x_3,x_4 \geq 0$    
  \par}

If the new variable $x_7$ is added to his problem with a column $
\begin{bmatrix}
  10\\10
\end{bmatrix}
$ %
and $c_7 = 120$ find the change in the optimal solution.
  \end{frameExample}


%%% Local Variables:
%%% mode: latex
%%% TeX-master: "slides_sensitivity"
%%% End:
