
\section{Changes in the Cost/Profit Coefficients $c_j$}
\label{sec:changes-cost-profit}

\begin{frame}{Changes in the Cost/Profit Coefficients $c_j$}
  \begin{itemize}\justifying \parskip3mm
    \item Changes in the coefficiets of the objective function may take place due to a change in cost or profit of either \alert{basic variables or non-basic variables}.
    \item Each of these two cases will first be considered separately. The discussion, be followed by a combined case.
  \end{itemize}
\end{frame}

\begin{frameExample}{\label{example:6.6-2.1}}{}
  % Example 6.6-2.1
  \begin{onlyenv}<1>
      A company wants to produce three products: $A, B, C$. The unit profits on these products are \$ 4, 6, and 2 respectively. These products require two types of resources -- manpower and material. The following L.P. model is formulated for determining the optimal product mix

  \[\max Z = 4x_1 +6x_2 +2x_3 \]

  {\centering
    subject to

    \vspace{3mm}
    \sysalign{r,r}%
    \sysdelim..%
  \systeme[x_1x_2x_3]%
  {
    x_1 + x_2 + x_3 \leq 3@(manpower),
    x_1 +4x_2 +7x_3 \leq 9@ (material)
  }

  $x_1, x_2, x_3 \geq 0$
  \par}

  Where   $x_1, x_2, x_3$ are the number of products $A, B $ and $C$ produced.
  \end{onlyenv}

\begin{onlyenv}<2>
  \begin{enumerate}[a)] \justifying \parskip3mm
  \item Find the optimal product mix and the corresponding profit to the company.
  \item Find the range on the values of non-basic variable coefficient $c_3$ such that the current optimal product mix remains optimal.
  \item What happens if $c_3$ is increased to \$ 12? What is the new optimal product mix in this case?
  \item Find the range on basic variable coefficient $c_1$ such that the current optimal product mix remains optimal.
  \item Find the effect when $c_1 = \$8$ on the optimal product mix.
  \item Find the effect of changing the objective function to $Z = 2x_1 + 8x_2 + 4x_3$ on the current optimal product mix.
  \end{enumerate}
\end{onlyenv}
\end{frameExample}



\begin{frameExample}{}{}

      \begin{columns}
    \column{0.5\textwidth}
    Given the L.P. Problem:

    $\max Z = -x_1 + 2x_2 - x_3$

    
    s.t.

    \vspace{5mm}
    \sysalign{r,r}%
    \sysdelim..%
    \systeme[x_1x_2x_3]%
    {
      3x_1 + x_2 - x_3 \leq 10,
      -x_1 + 4x_2 + x_3 \geq 6,
      x_2 + x_3 \leq 4 
    }

    \vspace{5mm}

    $x_1,x_2,x_3 \geq 0$
    \column{0.5\textwidth}

    determine the effect of discrete changes in $c_j (j = 1,2,\ldots,6)$ on the optimal basic feasible solution.
  \end{columns}

\end{frameExample}

\begin{frameExample}{}{}

  \begin{columns}[t]
    \column{0.5\textwidth}
    Consider the L.P. Problem:

    \[ \max Z = 3x_1 + 5x_2 + 4x_3\]
    
    \begin{equation*}
          \text{s.t.}
      \sysalign{r,r}%
    %\sysdelim..%
    \systeme[x_1x_2x_3]%
    {
      2x_1 + 3x_2  \leq 8,
      2x_2 + 5x_3 \leq 10,
      3x_1 + 2x_2 + 4x_3 \leq 15 
    }
    \end{equation*}

    \vspace{5mm}

    $x_1,x_2,x_3 \geq 0$
    \column{0.5\textwidth}
    \begin{enumerate}[a)] \parskip3mm \justifying
    \item How much $c_3$ and $c_4$ can be increased till the optimal solution ceases to be optimal? Also find the new value of the objective function if possible.
    \item Find the range over which $b_2$ can be changed maintaining the feasibility of the solution.
    \end{enumerate}
  \end{columns}
\end{frameExample}

%%% Local Variables:
%%% mode: latex
%%% TeX-master: "slides"
%%% End:
