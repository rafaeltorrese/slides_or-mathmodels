
\section{Changes in the Cost/Profit Coefficients $c_j$}
\label{sec:changes-cost-profit}

\begin{frame}{Changes in the Cost/Profit Coefficients $c_j$}
  \begin{itemize}\justifying \parskip3mm
    \item Changes in the coefficiets of the objective function may take place due to a change in cost or profit of either \alert{basic variables or non-basic variables}.
    \item Each of these two cases will first be considered separately. The discussion, be followed by a combined case.
  \end{itemize}
\end{frame}

\begin{frameExample}{\label{example:6.6-2.1}}{}
  % Example 6.6-2.1
  \begin{onlyenv}<1>
      A company wants to produce three products: $A, B, C$. The unit profits on these products are \$ 4, 6, and 2 respectively. These products require two types of resources -- manpower and material. The following L.P. model is formulated for determining the optimal product mix

  \[\max Z = 4x_1 +6x_2 +2x_3 \]

  {\centering
    subject to

    \vspace{3mm}
    \sysalign{r,r}%
    \sysdelim..%
  \systeme[x_1x_2x_3]%
  {
    x_1 + x_2 + x_3 \leq 3@(manpower),
    x_1 +4x_2 +7x_3 \leq 9@ (material)
  }

  $x_1, x_2, x_3 \geq 0$
  \par}

  Where   $x_1, x_2, x_3$ are the number of products $A, B $ and $C$ produced.
  \end{onlyenv}

\begin{onlyenv}<2>
  \begin{enumerate}[a)] \justifying \parskip3mm
  \item Find the optimal product mix and the corresponding profit to the company.
  \item Find the range on the values of non-basic variable coefficient $c_3$ such that the current optimal product mix remains optimal.
  \item What happens if $c_3$ is increased to \$ 12? What is the new optimal product mix in this case?
  \item Find the range on basic variable coefficient $c_1$ such that the current optimal product mix remains optimal.
  \item Find the effect when $c_1 = \$8$ on the optimal product mix.
  \item Find the effect of changing the objective function to $Z = 2x_1 + 8x_2 + 4x_3$ on the current optimal product mix.
  \end{enumerate}
\end{onlyenv}

\begin{onlyenv}<3>
  La tabla óptima del problema la encuentran \href{https://docs.google.com/spreadsheets/d/1HxunHqs_xUDkxKxFEE2Btu5kVv5buHvV2NC5ODKJBQA/edit?usp=sharing}{en este enlace: example-03}.

  \begin{table}[!ht]
    \centering
    \begin{tabular}{lc|rrrrr|r}
    \toprule
      ~ &$\max$ & 4 & 6 & 2 & 0 & 0 & ~ \\
      \midrule
      $c_{\boldsymbol{B}}$ & basis & x1 & x2 & x3 & s1 & s2 & $\boldsymbol{b}$ \\
      \midrule
        4 & x1 & 1 & 0 & -1 &  4/3  &  1/3  & 1 \\ 
      6 & x2 & 0 & 1 & 2 &  1/3  &  1/3  & 2 \\
      \midrule
        ~ & zj & 4 & 6 & 8 & 10/3  &  2/3  & 16 \\ 
        ~ & cj - zj & 0 & 0 & -6 & -10/3  &  2/3 \\ 
    \end{tabular}
\end{table}
  
\end{onlyenv}
\end{frameExample}



\begin{frameExample}{}{}

  \begin{onlyenv}<1>
          \begin{columns}
    \column{0.5\textwidth}
    Given the L.P. Problem:

    $\max Z = -x_1 + 2x_2 - x_3$

    
    s.t.

    \vspace{5mm}
    \sysalign{r,r}%
    \sysdelim..%
    \systeme[x_1x_2x_3]%
    {
      3x_1 + x_2 - x_3 \leq 10,
      -x_1 + 4x_2 + x_3 \geq 6,
      x_2 + x_3 \leq 4 
    }

    \vspace{5mm}

    $x_1,x_2,x_3 \geq 0$
    \column{0.5\textwidth}

    determine the effect of discrete changes in $c_j (j = 1,2,\ldots,6)$ on the optimal basic feasible solution.
  \end{columns}
  \end{onlyenv}

  \begin{onlyenv}<2>
    La tabla óptima la pueden consultar en \href{https://docs.google.com/spreadsheets/d/1rYsOiQEihozZusyn8mMk6Czqx2KFfgoRonP8k8F_Qsk/edit?usp=sharing}{el siguiente enlace: example-04}.

\begin{table}[!ht]
    \centering
    \begin{tabular}{ll|rrrrrrr|r}
\toprule    
      $\max$ & cj & -1 & 2 & -1 & 0 & 0 & 0 & -1000 & ~ \\
      \midrule
      $c_{\boldsymbol{B}}$ & basis & x1 & x2 & x3 & s1 & s2 & s3 & A2 & $\boldsymbol{b}$ \\
      \midrule
        0 & s1 & 3 & 0 & -2 & 1 & 0 & -1 & 0 & 6 \\ 
      2 & x2 & 0 & 1 & 1 & 0 & 0 & 1 & 0 & 4 \\
      0 & s2 & 1 & 0 & 3 & 0 & 1 & 4 & -1 & 10 \\
            \midrule
        ~ & zj & 0 & 2 & 2 & 0 & 0 & 2 & 0 & 8 \\ 
      ~ & cj - zj & -1 & 0 & -3 & 0 & 0 & -2 & -1000 \\
      \bottomrule
    \end{tabular}
\end{table}
  \end{onlyenv}
\end{frameExample}

\begin{frameExample}{}{}

  \begin{onlyenv}<1>
      \begin{columns}[t]
    \column{0.5\textwidth}
    Consider the L.P. Problem:

    \[ \max Z = 3x_1 + 5x_2 + 4x_3\]
    
    \begin{equation*}
          \text{s.t.}
      \sysalign{r,r}%
    %\sysdelim..%
    \systeme[x_1x_2x_3]%
    {
      2x_1 + 3x_2  \leq 8,
      2x_2 + 5x_3 \leq 10,
      3x_1 + 2x_2 + 4x_3 \leq 15 
    }
    \end{equation*}

    \vspace{5mm}

    $x_1,x_2,x_3 \geq 0$
    \column{0.5\textwidth}
    \begin{enumerate}[a)] \parskip3mm \justifying
    \item How much $c_3$ and $c_4$ can be increased till the optimal solution ceases to be optimal? Also find the new value of the objective function if possible.
    \item Find the range over which $b_2$ can be changed maintaining the feasibility of the solution.
    \end{enumerate}
  \end{columns}
  \end{onlyenv}

  \begin{onlyenv}<2>

    La tabla óptima se muestra a continuación, pueden consultar el resultado \href{https://docs.google.com/spreadsheets/d/12gZfxcK0EZ0_j0RBfGCaoFGxBb17jhfYmBv8LngZivY/edit?usp=sharing}{ en este enlace}
    \begin{table}[!ht]
    \centering
    \begin{tabular}{cc|rrrrrr|r}
            ~ & ~ & $c_1$ & $c_2$ & $c_3$ & $c_4$ & $c_5$ & $c_6$ & ~ \\
      \toprule
      $\max$ & $c_j$ & 3 & 5 & 4 & 0 & 0 & 0 & ~ \\
      \midrule
      $c_{\boldsymbol{B}}$ & basis & $x_1$ & $x_2$ & $x_3$ & $s_1$ & $s_2$ & $s_3$ & $\boldsymbol{b}$ \\
      \midrule
        5 & $x_2$ &   0   & 1 & 0 &  15/41  &   8/41  &  -10/41  &  50/41  \\ 
        4 & $x_3$ &   0   & 0 & 1 &   -6/41  &   5/41  &   4/41  &  62/41  \\ 
      3 & $x_1$ & 1 & 0 & 0 &   -2/41  &  -12/41  &  15/41  &  89/41  \\
      \midrule
        ~ & $z_j$ & 3 & 5 & 4 &  45/41  &  24/41  &  11/41  & 765/41  \\ 
      ~ & $c_j - z_j$ & 0 & 0 & 0 & - 45/41  &  -24/41  &  -11/41 \\
      \bottomrule
    \end{tabular}
  \end{table}

  Las columnas $c_3$ y $c_4$ corresponden a las variables $x_3$ y $s_1$ de las cuales solo $x_3$ es una variable básica.
\end{onlyenv}

\begin{onlyenv}<3>
  ¿Cuánto $c_3$ y $c_4$ pueden aumentar hasta que la solución óptima deje de ser óptima? También encuentra el nuevo valor de la función objetivo si es posible.

  El coeficiente $c_3$ pertence a la variable básica $x_3 = 4$. Se quiere maximizar el valor de la función objectivo por lo tanto toda la fila $\bar{c_j} = c_j - z_j$ debe cumplir $\leq 0$ por lo tanto para las variables \alert{no básicas} se debe cumplir $ \bar{c_j} = c_j - z_j \leq 0$ por ser un problema de maximización. Para $c_4, c_5, c_6$ se tiene

  \begin{columns}
    \column{0.4\textwidth}
    $\overline{c_4} = c_4 - z_4 \leq 0$

    \begin{align*}
    \overline{c_4} &= 0 - \begin{bmatrix} 5 & c_3 & 3 \end{bmatrix}
    \begin{bmatrix}
      \nicefrac{15}{41}\\ \nicefrac{-6}{41}\\ \nicefrac{-2}{41}\\
    \end{bmatrix}
      \leq 0 \\
      \overline{c_4} &= 0 - \left(\nicefrac{75}{41} - \nicefrac{6}{41}c_3 - \nicefrac{6}{41}\right) \leq 0
    \end{align*}
    
    \column{0.3\textwidth}
    $\overline{c_5} = c_5 - z_5 \leq 0$
    \column{0.3\textwidth}
    $\overline{c_6} = c_6 - z_6 \leq 0$
  \end{columns}
  
\end{onlyenv}
\end{frameExample}

%%% Local Variables:
%%% mode: latex
%%% TeX-master: "slides"
%%% End:
