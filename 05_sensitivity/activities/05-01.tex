\documentclass[spanish,letterpaper,11pt]{exam}


% -------------------- Packages --------------------
\usepackage{amsmath}
\usepackage{amssymb}
\usepackage{booktabs}
\usepackage{systeme}
\usepackage{cancel}
\usepackage[shortlabels]{enumitem}
% -------------------- Definitions --------------------

\newcommand{\tema}{Análisis de Sensibilidad}
%
\extrafootheight{-0.5in}
\header%
{Dr. Rafael Torres Escobar}% Left
{INVESTIGACIÓN DE OPERACIONES -- MODELOS MATEMÁTICOS \\ \tema}% Center
{Página \\ \thepage\ de \numpages} % Right
\headrule
\pointpoints{punto}{puntos}
\author{Dr. Rafael Torres Escobar}
%
%\printanswers % comentar para mostrar respuestas
% ================================================== 
\begin{document}
\begin{questions}
    \question
    Dado el siguiente modelo de programación lineal

    \[ \max Z = -x_1 + 2x_2 - x_3 \]
  
{\centering
\vspace{2mm}
s.t.
\vspace{2mm}

\sysalign{r,r}%
\sysdelim..%
\systeme[x_1x_2x_3]%
{
  3x_1 + x_2 - x_3 \leq 10,
  -x_1 + 4x_2 + x_3 \geq 6,
  x_2 + x_3 \leq 4
}

\vspace{3mm}
$x_1, x_2, x_3 \geq 0$
\par}
  
\begin{enumerate}[a)] 
    \item Encuentre la solución óptima.
    \item Encuentre los rangos separados $b_1, b_2 $ y $b_3$ consistentes con la solución óptima.
\end{enumerate}

  
\vspace{6mm}

  \question
  Dado el siguiente modelo de P.L.:

  \[ \max Z = -x_1 + 2x_2 - x_3 \]

  
{\centering
\vspace{2mm}
s.t.
\vspace{2mm}

\sysalign{r,r}%
\sysdelim..%
\systeme[x_1x_2x_3]%
{
  3x_1 + x_2 - x_3 \leq 10,
  -x_1 + 4x_2 + x_3 \geq 6,
  x_2 + x_3 \leq 4 
}

\vspace{3mm}

$x_1,x_2,x_3 \geq 0$
\par}


\begin{enumerate}[a)]
    \item Determine el efecto de cambios discretos en $c_j \, (j = 1,2,\ldots,6)$ en la solución básica factible óptima.
\end{enumerate}

\end{questions}
\end{document}