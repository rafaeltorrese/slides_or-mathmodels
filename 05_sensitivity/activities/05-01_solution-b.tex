\begin{solution}
  
  Los rangos se determinan multiplicando la matriz inversa $B^{-1}$ por el vector $b + \Delta_i$ para cada $i = 1, 2, m$ en donde $m$ es el número de ecuaciones en formato estándar. Esto se expresa de la siguiente manera:

  \[ \boldsymbol{B^{-1}} (\boldsymbol{b + \Delta}) \geq \boldsymbol{0}\]

  Para cada $\Delta_i$ tenemos los siguiente los rangos $b_i$ se determinan de la siguiente forma

  \begin{align*}
    b_1 &= %
          \begin{bmatrix}
      1& 0& -1\\
      0& 0& 1\\
      0& -1& 4\\
    \end{bmatrix}
    \begin{bmatrix}
      10 + \Delta_1\\
    6\\
    4\\
  \end{bmatrix}\geq
      \begin{bmatrix}
      0\\
    0\\
    0\\
  \end{bmatrix}
    \\
    b_2 & = %
          \begin{bmatrix}
      1& 0& -1\\
      0& 0& 1\\
      0& -1& 4\\
    \end{bmatrix}
    \begin{bmatrix}
      10 \\
    6 + \Delta_2\\
    4\\
  \end{bmatrix}\geq
      \begin{bmatrix}
      0\\
    0\\
    0\\
  \end{bmatrix}
    \\
        b_3 & = %
          \begin{bmatrix}
      1& 0& -1\\
      0& 0& 1\\
      0& -1& 4\\
    \end{bmatrix}
    \begin{bmatrix}
      10 \\
    6 \\
    4 + \Delta_3\\
  \end{bmatrix}\geq
      \begin{bmatrix}
      0\\
    0\\
    0\\
  \end{bmatrix}    
  \end{align*}
  
  Al realizar la multiplicación de matrices para cada $\Delta_i$ tenemos lo siguiente

  
  \begin{flalign*}
    \Delta_1 & = %
    \begin{bmatrix}
      6 + \Delta_1\\
      4\\
      10 \\
    \end{bmatrix}\geq
    \begin{bmatrix}
      0\\
    0\\
    0\\
  \end{bmatrix}\\
  %
  %
  \Delta_2 & = %
    \begin{bmatrix}
      6 \\
      4\\
      10 - \Delta_2\\
    \end{bmatrix}\geq
    \begin{bmatrix}
      0\\
    0\\
    0\\
  \end{bmatrix}\\
  %
  %
      \Delta_3 & = %
    \begin{bmatrix}
      6 - \Delta_3\\
      4+ \Delta_3\\
      10 + 4\Delta_3\\
    \end{bmatrix}\geq
    \begin{bmatrix}
      0\\
    0\\
    0\\
  \end{bmatrix}
  %
  %
  \end{flalign*}

  Para encontrar el primer rango $\Delta_1$ tenemos que 
  \begin{flalign*}
    6 + \Delta_1 & \geq 0\\
    4 & \geq 0\\
    10 & \geq 0\\
  \end{flalign*}

  por lo tanto tenemos que $\Delta_1 \geq -6$ así el rango es \[-6 \leq \Delta_1 \leq +\infty \] y sustituyendo el valor de $\Delta_1$ en $b_1$ encontramos que el rango para $b_1$ es

  \begin{align*}
    10 - 6 & \leq b_1 \leq 10 + \infty\\[2mm]
    4 & \leq b_1 \leq +\infty
  \end{align*}

  %% Delta2
    Para encontrar $\Delta_2$ hacemos lo mismo 
  \begin{flalign*}
    6  & \geq 0\\
    4 & \geq 0\\
    10 - \Delta_2& \geq 0\\
  \end{flalign*}

  por lo tanto tenemos $\Delta_2 \leq 10$ y su rango es \[-\infty \leq \Delta_2 \leq 10 \] y sustituyendo el valor de $\Delta_2$ en $b_2$ encontramos que el rango para $b_2$ es

  \begin{align*}
    6 - \infty & \leq b_2 \leq 6 + 10\\[2mm]
    0 & \leq b_2 \leq 16
  \end{align*}


    %% Delta3
    Para encontrar $\Delta_3$ tenemos lo siguiente
  \begin{flalign*}
    \Delta_3  & \leq 6\\
    \Delta_3 & \geq -4\\
    \Delta_3& \geq -\frac{5}{2}\\
  \end{flalign*}

  por lo tanto tenemos $\Delta_3 \geq -\frac{5}{2}$ y $\Delta_3 \leq 6$, su rango es \[-\frac{5}{2} \leq \Delta_3 \leq  6\] y sustituyendo el valor de $\Delta_3$ en $b_3$ encontramos que el rango para $b_3$ es

  \begin{align*}
    4 - \frac{5}{2} & \leq b_3 \leq 4 + 6\\[2mm]
    \frac{3}{2} & \leq b_3 \leq 10
  \end{align*}
\end{solution}



%%% Local Variables:
%%% mode: latex
%%% TeX-master: "05-01"
%%% End:
