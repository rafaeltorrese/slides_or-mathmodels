\begin{solution}

  El problema es su forma estándar es

  \[ \max Z = -x_1 + 2x_2 - x_3 + 0s_1 - 0s_2 + 0s_3 - MA_2\]
  
{\centering
\vspace{2mm}
s.t.
\vspace{2mm}

\sysalign{r,r}%
\sysdelim..%
\systeme[x_1x_2x_3s_1s_2s_3A_2]%
{
  3x_1 + x_2 - x_3 + s_1 = 10,
  -x_1 + 4x_2 + x_3 - s_2 + A_2= 6,
  x_2 + x_3 + s_3 = 4
}

\vspace{4mm}
$x_1, x_2, x_3, s_1, s_2, s_3 \geq 0$
\par}

La tabla simplex inicial a partir del formato estándar es 

{    \centering
    \begin{tabular}{rcrrrrrrrr}
\toprule    
        $\max$ & $c_j$ & -1 & 2 & -1 & 0 & 0 & 0 & -1000 & ~ \\ \midrule
        $c_{\boldsymbol{B}}$ & basis & $x_1$ & $x_2$ & $x_3$ & $s_1$ & $s_2$ & $s_3$ & $A_2$ & $\boldsymbol{b}$ \\ \midrule
        0 & $s_1$ &  3 & 1 & -1 & 1 &  0 & 0 &  0 & 10 \\ 
    -1000 & $A_2$ & -1 & 4 &  1 & 0 & -1 & 0 &  1 &  6\\ 
        0 & $s_3$ &  0 & 1 &  1 & 0 &  0 & 1 &  0 &  4 \\ \midrule
        ~ & $z_j$ & 1000 & -4000 & -1000 & 0 & 1000 & 0 & -1000 & -6000 \\ 
        ~ & $c_j - z_j$ & -1001 & 4002 & 999 & 0 & -1000 & 0 & 0 \\ \toprule
    \end{tabular} \par}


\vspace{1cm}
  La tabla óptima del problema se muestra a continuación 

 
{    \centering
    \begin{tabular}{lcrrrrrrrr}
\toprule    
        $\max$ & $c_j$ & -1 & 2 & -1 & 0 & 0 & 0 & -1000 & ~ \\ \midrule
        $c_{\boldsymbol{B}}$ & basis & $x_1$ & $x_2$ & $x_3$ & $s_1$ & $s_2$ & $s_3$ & $A_2$ & $\boldsymbol{b}$ \\ \midrule
        0 & $s_1$ & 3 & 0 & -2 & 1 & 0 & -1 & 0 & 6 \\ 
        2 & $x_2$ & 0 & 1 & 1 & 0 & 0 & 1 & 0 & 4 \\ 
        0 & $s_2$ & 1 & 0 & 3 & 0 & 1 & 4 & -1 & 10 \\ \midrule
        ~ & $z_j$ & 0 & 2 & 2 & 0 & 0 & 2 & 0 & 8 \\ 
        ~ & $c_j - z_j$ & -1 & 0 & -3 & 0 & 0 & -2 & -1000 \\ \toprule
    \end{tabular} \par}

  De acuerdo a la tabla óptima, $s_1, x_2$ y $s_2$ forman la base del problema (ver columna \textbf{basis}). 

  La matriz base se forma con los coeficientes $s_1, x_2$ y $s_2$ del problema, lo que nos da la siguiente matriz

  $\boldsymbol{B} = %
  \begin{bmatrix}
    s1& x2& s2\\
    1& 1& 0\\
    0& 4& -1\\
    0& 1& 0\\
  \end{bmatrix}
  $  

  Para calcular una solución al problema se toma la matriz inversa de la matriz base $\boldsymbol{B^{-1}}$ y se multiplica por el nuevo vector $\boldsymbol{b}$, esto es, $\boldsymbol{B^{-1}}\cdot \boldsymbol{b}$. La matriz $\boldsymbol{B^{-1}}$ es

  \[ \boldsymbol{B^{-1}} =%
    \begin{bmatrix}
    1& 1& 0\\
    0& 4& -1\\
    0& 1& 0\\
  \end{bmatrix}^{-1} =%
    \begin{bmatrix}
      1& 0& -1\\
      0& 0& 1\\
      0& -1& 4\\
    \end{bmatrix}
  \]

  
  De esta manera si queremos saber el rango de cada uno de los lados derecho del probema, para cada elemento en el vector de lado derecho $\boldsymbol{b}$ se suma o resta una incremento $\Delta$. La suma o resta de una $\Delta$ depende del valor de las variables duales en la tabla óptima. El valor
  de las variables duales en el problema se pueden obtener de la fila $z_j$

  {    \centering
    \begin{tabular}{lcrrrrrrrr}
      ~ & ~ & $x_1$ & $x_2$ & $x_3$ & $s_1$ & $s_2$ & $s_3$ & $A_2$ & $\boldsymbol{b}$ \\ \midrule
\toprule
          ~ & $z_j$ & 0 & 2 & 2 & \cellcolor{yellow}0 & \cellcolor{yellow}0 & \cellcolor{yellow}2 & 0 & \cellcolor{green}8 \\ 
      ~ & $c_j - z_j$ & -1 & 0 & -3 & 0 & 0 & -2 & -1000 \\ \toprule
    \end{tabular}\par}

  En la fila $z_j$ de la tabla óptima vemos que las columnas correspondientes a $s_1, s_2, s_3$ son los valores de las variables duales $y_!, y_2, y_3$ y todas son positivas, esto quiere decir que para determinar los rangos $b_1, b_2, b_3$ se suman incrementos $\Delta_1, \Delta_2, \Delta_3$ respectivamente.
 
\end{solution}



%%% Local Variables:
%%% mode: latex
%%% TeX-master: "05-01"
%%% End:
