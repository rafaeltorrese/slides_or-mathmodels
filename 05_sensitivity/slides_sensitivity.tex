\documentclass[../slides.tex]{subfiles}
\graphicspath{{figs/}}
\title{Sensitivity Analysis}

\AtBeginSection[] % Do nothing for \section* %
{
\begin{frame}<beamer> 
  \frametitle{Agenda}
  \tableofcontents[currentsection] 
\end{frame}
}


\begin{document}
% ==============================
\begin{frame}
  \maketitle
\end{frame}


\begin{frame}{Agenda}
  \tableofcontents
\end{frame}
% ==============================

\section{What Is Sensitivity Analysis in L.P.?}
\label{sec:sensitivityintro}


\begin{frame}{Sensitivity Analysis}
  \begin{itemize} \justifying \parskip4mm
  \item<only@1> It is desirable to study how the current optimal solution changes when the parameters of the problem get changed. In these problems this inoformation may be more important than the single result provided by the optimal solution.
  \item<only@1> After attaining the optimal solution, one may discover that \alert{a wrong value of a cost coefficient was used or a particular variable or constratint was ommited} or one or more of tight-hand constants used were wrong.
  \item<only@1> The changes in parameters of the problem may be \alert{discrete or continuous}. The study of the effect of \alert{discrete changes in parameters on the optimal solution} is called the \alert{sensitivity analysis} or \alert{post optimality analysis}, while continuous changes in parameters is called \alert{parametric programming}.
  \item<only@2> The changes in the parameters of a linear programming problem include:%
    \begin{enumerate} \justifying
    \item Changes in the right-hand side of the constraints or availability of resources ($b_i$)
    \item Changes in the cost/profit coefficients or cost / profit contribution per unit of decision variables ($c_j$).
    \item Addition of new variables
    \item Changes in the coefficients of constraints or consumption of resources per unit of decision variables $a_{ij}$.
    \item Addition of new constraints.
    \item Deletion of variables.
    \item Deletion of constraints.
    \end{enumerate}
  \end{itemize}
\end{frame}


\section{Changes in The Right-Hand Side}
\label{sec:changes-right-hand-side}

\begin{frame}{Changes in The Right-Hand Side of The Constraints $b_i$}{}
  \begin{itemize} \justifying \parskip3mm
  \item Suppose that an optimal solution to a linear programming has already been found and it is desired to find the effect of increasing or decreasing some resource.
  \item This will affect not only the objective function but also the solution.
  \item Large changes in the limiting resources \alert{may even change the variables in the solution since one or more current basic variables become negative}.
  \item Dual simplex method is used to remove infeasibility and to get a feasbile optimal solution.
  \end{itemize}
  
\end{frame}

\begin{frameExample}{}{}

  \begin{columns}
    \column{0.4\textwidth}
      $\max Z = 5x_1 + 12x_2 + 4x_3$

      \vspace{5mm}

      s.t.
  \sysalign{r,r}%
  \systeme[x_1x_2x_3]{%
    x_1 + 2x_2 + x_3 \leq 5,
    2x_1 - x_2 +3x_3 = 2  
  }

  \vspace{5mm}
  
  $x_1 , x_2, x_3 \geq 0$
  \column{0.6\textwidth}
  \begin{enumerate}[a)] \justifying \parskip4mm
  \item Solve the problem.
  \item Discuss the effect of changing the requirement vector from $%
    \begin{bmatrix}
      5\\
      2\\
    \end{bmatrix}
    $%
    to%
    $
    \begin{bmatrix}
      7\\
      2\\
    \end{bmatrix}
$
\item Discuss the effect of changing the requirement vector from %
  $
  \begin{bmatrix}
    5\\2\\
  \end{bmatrix}
  $
  to%
  $
  \begin{bmatrix}
    3\\9\\
  \end{bmatrix}
  $
  \item Which resource should be increased and how much to achieve the best marginal increase in the value of the objective function?
  \end{enumerate}
  \end{columns}
\end{frameExample}


\begin{frameExample}{}{}
\begin{columns}
  \column{0.5\textwidth}
  Given the following L.P. Problem:

  \[ \max Z = -x_1 + 2x_2 - x_3 \]

  \vspace{6mm}
  s.t.
  \sysalign{r,r}%
  \systeme[x_1x_2x_3]%
  {
    3x_1 + x_2 - x_3 \leq 10,
    -x_1 + 4x_2 + x_3 \geq 6,
    x_2 + x_3 \leq 4
  }

  \vspace{6mm}

  $x_1, x_2, x_3 \geq 0$
  \column{0.5\textwidth}
  \begin{enumerate}[a)] \justifying \parskip3mm
  \item Find the optimal solution.
  \item   Find the separate ranges $b_1, b_2 $ and $b_3$ consistent with the optimal solution.
  \end{enumerate}
\end{columns}  
\end{frameExample}


%%% Local Variables:
%%% mode: latex
%%% TeX-master: "slides_sensitivity"
%%% End:




\end{document}
