
\section{Duality}
\label{sec:duality}



\begin{frame}{Definición del Problema Dual}
  \begin{itemize}\justifying \parskip3mm


\item  El problema dual se define sistemáticamente a partir del modelo de PL primal (u original). Los dos problemas están estrechamente relacionados en el sentido de que la solución óptima de uno proporciona automáticamente la solución óptima al otro.
\item   En la mayoría de los tratamientos de PL, el dual se define para varias formas del primal según el sentido de la optimización (maximización o minimización), los tipos de restricciones ($\leq,  \geq, =$), y el signo de las variables (no negativas o irrestrictas).
  \end{itemize}
\end{frame}

\begin{frame}{}{}
  \only<1>{Las ideas clave para construir el dual a partir del primal se resumen como sigue:}
  
  \begin{enumerate} \justifying \parskip4mm
\item<only@1> Asigne una variable dual por cada restricción primal.
\item<only@1> Construya una restricción dual por cada variable primal.
\item<only@1> Los coeficientes de restricción (columna) y el coeficiente objetivo de la variable
primal $j$-ésima definen respectivamente los lados izquierdo y derecho de la restricción dual $j$-ésima.
\item<only@2>  Los coeficientes objetivo duales son iguales a los lados derechos de las ecuaciones de restricción primales.
\item<only@2> Las reglas que aparecen en la tabla siguiente rigen el sentido de optimización, la dirección de las desigualdades y los signos de las variables en el dual. Una forma fácil
de recordar el tipo de restricción en el dual (es decir, $\leq$ o $\geq$) es que si el objetivo
dual es de \alert{minimización} (es decir, apunta hacia abajo), entonces todas las restricciones serán del tipo $\geq$ (es decir, apuntan hacia arriba). Lo opuesto aplica cuando el objetivo dual es de \alert{maximización}.
\end{enumerate}

\begin{onlyenv}<3>
  {\centering
\begin{tabular}{ccccccc}
 & $x_1$ & $x_2$ & $x_3$ & $\cdots $ & $x_n$ &  \\\cline{2-6}
\multicolumn{1}{l|}{$y_1$} & $a_{11}$ & $a_{12}$ & $a_{13}$ & $\cdots$ & \multicolumn{1}{l|}{$a_{1n}$} & $\leq b_1$ \\
\multicolumn{1}{l|}{$y_2$} & $a_{21}$ & $a_{22}$ & $a_{23}$ & $\cdots$ & \multicolumn{1}{l|}{$a_{22n}$} & $\leq b_2$ \\
\multicolumn{1}{l|}{$y_3$} & $a_{31}$ & $a_{32}$ & $a_{33}$ & $\cdots$ & \multicolumn{1}{l|}{$a_{3n}$} & $\leq b_3$ \\
\multicolumn{1}{l|}{$\vdots$} & $\vdots$ & $\vdots$ & $\vdots$ & $\vdots$ & \multicolumn{1}{l|}{$\vdots$} & $\vdots$ \\
\multicolumn{1}{l|}{$y_m$} & $a_{m1}$ & $a_{m2}$ & $a_{m3}$ & $\cdots$ & \multicolumn{1}{l|}{$a_{mn}$} & $\leq b_m$ \\ \cline{2-6}
 & $\geq c_1$ & $\geq c_2$ & $\geq c_3$ & $\cdots$ & $\geq c_n$ & 
\end{tabular}
  \par}
\end{onlyenv}
\end{frame}

\begin{frameExample}{Primal - Dual}{}
  \begin{columns}
    \column{0.5\textwidth}
        \begin{align*}
      \max Z = 3x_1 + 5x_2 & \\[3mm]
    2x_1 + 6x_2 & \leq 50\\
    3x_1 + 2x_2 & \leq 35\\
    5x_1 - 3x_2 & \leq 10\\
    x_2 & \leq 20\\[5mm]
    x_1, x_2 & \geq 0
        \end{align*}
        
    \column{0.5\textwidth}
                 \[      \min W = 50y_1 + 35y_2  + 10y_3 + 20y_4  \]
                 s.t.
                 \vspace{3mm}

                 \sysdelim..%
                 \sysalign{r,r}%                 
                 \systeme[y_1y_2y_3y_4]{%
    2y_1 + 3y_2 + 5y_3  \geq 3,
    6y_1 + 2y_2 - 3y_3 + y_4 \geq 5
  } % systeme

  \vspace{5mm}
  $    y_1, y_2, y_3, y_4  \geq 0$
  \end{columns}
\end{frameExample} 


\begin{frameExample}{Primal - Dual}{}
  % EXAMPLE 6.1-1.2
En el problema primal, convertir desigualdades $\leq$ a desigualdades $\geq$ cuando se busca \alert{minimizar} la función objetivo, y viceversa.
  \begin{columns}
    \column{0.5\textwidth}
\[      \min Z = 3x_1 - 2x_2 + 4x_3  \]
s.t. \vspace{4mm}

\sysalign{r,r}\sysdelim..\systeme[x_1x_2x_3]{%
      3x_1 + 5x_2 + 4x_3 \geq 7,
    6x_1 + x_2 + 3x_3  \geq 4,
    7x_1 - 2x_2 - x_3  \leq 10@\star,
    x_1 - 2x_2 + 5x_3  \geq 3,
    4x_1 + 7x_2 -2x_3  \geq 2
}% systeme

          $    x_1, x_2, x_3  \geq 0$
    \column{0.5\textwidth}
    \[      \max W = 7y_1 + 4y_2  - 10y_3 + 3y_4 + 2y_5  \]
    
    s.t.
    \vspace{4mm}

    \sysalign{r,r}\sysdelim..\systeme[y_1y_2y_3y_4y_5]{%
    3y_1 + 6y_2 - 7y_3 + y_4 + 4y_5  <= 3,
5y_1 + y_2 + 2y_3 - 2y_4 + 7y_5  <= -2,
4y_1 + 3y_2 + y_3 + 5y_4 - 2y_5  <= 4
  } % systeme

  \vspace{5mm}
  $    y_1, y_2, y_3, y_4, y_5  \geq 0$
  \end{columns}
\end{frameExample}


\begin{frameExample}{Primal - Dual}{}
  % EXAMPLE 6.1-1.3
  $  \max Z = 3x_1+ 17x_2 + 9x_3 $
  
  s.t.\vspace{5mm}


  \sysalign{r,r}\sysdelim..\systeme[x_1x_2x_3]{%
x_1 - x_2 + x_3  >= 3,
-3x_1 + 2x_3 <= 1
}
$\quad \Leftrightarrow \quad$ \systeme[x_1x_2x_3]{%
-x_1 + x_2 - x_3  <= -3,
-3x_1 + 2x_3 <= 1
}

$x_1, x_2, x_3  \geq 0 $
\vspace{5mm}

\begin{block}{Dual}
  
  $\min W = -3y_1 + y_2 $

                  s.t. \sysalign{r,r}\systeme[y_1y_2]{%
    -y_1 - 3y_2 >= 3,
y_1 >= 17,
-y_1 + 2y_2 >= 9
  } % systeme
  
  $    y_1, y_2  \geq 0 $
\end{block}
\end{frameExample}


\section{Dual Transformation for L.P.P. in Standard Form}
\label{sec:standard-form}


\begin{frameExample}{Primal - Dual}{}
  % EXAMPLE 6.1-2.1
   $  \max Z = 3x_1+ 10x_2 + 2x_3 $
  
  s.t.\sysalign{r,r}\systeme[x_1x_2x_3]{%
2x_1 + 3x_2 + 2x_3  <= 7,
3x_1 - 2x_2 + 4x_3 = 3
} $\quad \Leftrightarrow \quad$
\systeme[x_1x_2x_3]{%
2x_1 + 3x_2 + 2x_3  <= 7@y_1,
3x_1 - 2x_2 + 4x_3 <= 3@y_2^{\prime},
-3x_1 + 2x_2 - 4x_3 <= -3@y_2^{\prime\prime},
}

$x_1, x_2, x_3  \geq 0 $
\vspace{3mm}
\begin{block}{Dual}
  
  $\min W = 7y_1 + 3y_2 $

                  s.t. \sysalign{r,r}\systeme[y_1y_2]{%
                    2y_1 + 3y_2 >= 3,
                    3y_1 - 2y_2 >= 10,
                    2y_1 + 4y_2 >= 2
  } % systeme
  
  $    y_1\geq 0 \, , y_2 \text{ irrestricta en signo}$
\end{block}
\end{frameExample}


\begin{frameExample}{Dual con Igualdades en Problema Primal}
  % Example 6.1-2.2 Gupta
  
  $\min Z = x_2 + 3x_3$

  s.t.

  \sysalign{r,r}%
  \systeme[x_1x_2x_3]%
  {%
    2x_1 + x_2 \leq 3,
    x_1 + 2x_2 + 6x_3 \geq 5,
    -x_1 + x_2 + 2x_3 = 2,
  }

  $x_1, x_2, x_3 \geq 0$
\end{frameExample}

\begin{frameExample}{}
  % Example 6.1-2.3 Gupta

  $\max Z = 2x_1 + x_2 + x_3$

  s.t.

  \sysalign{r,r}%
  \systeme[x_1x_2x_3]%
  {%
    x_1 + x_2 + x_3 \geq 6,
    3x_1 - 2x_2 + 3x_3 = 3,
    -4x_1 + 3x_2 - 6x_3 = 1
  }

  $x_1, x_2, x_3 \geq 0$

  
\end{frameExample}

\begin{frameExample}{}
  % Example 6.1-2.4 Gupta

  $\min Z = 3x_1 - 2x_2 + x_3$

  s.t.

  \sysalign{r,r}%
  \systeme[x_1x_2x_3]%
  {%
   2x_1 - 3x_2 + x_3 \leq 5,
    4x_1 - 2x_2  \geq  9,
    -8x_1 + 4x_2 + 3x_3 = 8
  }

  $x_1, x_2\geq 0, \, x_3 \text{  is unrestricted} $

  
\end{frameExample}


%%% Local Variables:
%%% mode: latex
%%% TeX-master: "slides"
%%% End:
