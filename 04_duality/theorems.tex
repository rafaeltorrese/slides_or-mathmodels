
\section{Duality Theorems}
\label{sec:duality-theorems}

\begin{frame}{Duality Theorems}
  \begin{enumerate} \justifying \parskip4mm
\item<only@1> The dual of the dual is the primal.
\item<only@1> The value of the objective function $Z$ for any feasible solution of the primal is $\leq$ the value of the objective function $W$ for any feasible solution of the dual ( $Z \leq W$).
\item<only@1> If either the primal or the dual problem has an unbounded solution, then the solultion to the other problem is infeasible.
\item<only@1> \textbf{(Main Duality Theorem)}. If both the primal and the dual problems have feasible solutions, then both have optimal solutions and  $\max Z = \min W$
\item<only@1> Complementary Slackness.
\end{enumerate}

\begin{theorem}<only@2>[Complementary Slackness Theorem]
  \begin{enumerate} \justifying \parskip2mm
  \item If a \alert{primal variable is positive}, then the corresponding dual constraint is an equation at the optimum.
  \item If the \alert{primal constraint is a strict inequality}, then the corresponding dual variable
    is zero at the optimum.
    
  \item If a \alert{dual variable is positive}, then the corresponding constraint is an equation at the optimum.
  \item If a \alert{dual constraint is a strict inequality}, then the corresponding primal variable is zero at the optimum.
  \end{enumerate}
\end{theorem}
\end{frame}

\begin{frameExample}{}{}
  \[ \max Z = 2x_1 + x_2 \]
      {\centering
      s.t.%
\vspace{4mm}
      
        \sysdelim..%
        \sysalign{r,r}%
        \systeme[x_1x_2]%
        {%
          x_1 + 2x_2  \leq 10,
          x_1 + x_2  \leq 6,
          x_1 - x_2  \leq 2,
          x_1 - 2x_2  \leq 1
      }

      \vspace{5mm}
      $x_1, x_2 \geq 0$
      \par}
\end{frameExample}


\begin{frameExample}{}{}
  \[ \max Z = 5x_1 + 4x_2 \]
      {\centering
      s.t.%
\vspace{4mm}
      
        \sysdelim..%
        \sysalign{r,r}%
        \systeme[x_1x_2]%
        {%
          6x_1 + 4x_2  \leq 24,
          x_1 + 2x_2  \leq 6,
          -x_1 + x_2  \leq 1,
          x_2  \leq 2
      }

      \vspace{5mm}
      $x_1, x_2 \geq 0$
      \par}
\end{frameExample}
  
\begin{frame}{Correspondencia entre las Soluciones Primal--Dual}

  \begin{enumerate} \justifying \parskip3mm
  \item Valores de las variables no-básicas del primal se obtienen de la fila base en la solución dual, bajo las variables de holgura (si existen) ignorando el signo negativo y bajo las vraibles artificiales (si no hay variables de holgura en la restricción) ignorando el signo negativo después de borrar la constante $M$.
  \item Los valores de las variables de holgura del problema primal se obtienen de la fila base ($c_j - Z_j$) bajo las variables no-básicas en la solución dual ignorando el signo negativo.
  \item El valor de la función objetivo es el mismo en la solución primal y dual.
  \end{enumerate}
\end{frame}



\begin{frameExample}{}{}
  \only<1>{%
  A feed mixing operation can be described in terms of the two activities. The required mixture must contain four kinds of ingredients $w, x, y$ and $z$. Two basic feeds $A$ and $B$, which contain the required ingredients are available in the market. One kg. of $A$ contains 0.1 kg. of $w$, 0.1 kg. of $y$ and 0.2 kg. of $z$. Likewise, one kg. of feed $B$ contains 0.1 kg. of $x$, 0.2 kg. of $y$,  and 0.1 kg. of $z$. The daily per head requirement is of at least 0.4 kg. of $w$, 0.6 kg. of $x$, 2 kg. of $y$ and 1.8 kg. of $z$. Feed $A$ can be bought for \$0.07 per kg. and $B$ can be bought for \$ 0.05 per kg. The availabilities, requirements and costs are summarized in the following table.

  Determine the quantities of feeds $A$ and $B$ in the mixture so that the total cost is minimum.
  }%

  \begin{onlyenv}<2>
      {\centering
    \begin{tabular}{cccc}
      \toprule
      Ingredient&Feed $A$& Feed $B$& Requirement\\
      \midrule
      &(kg.)&(kg.)&(kg.)\\
$w$&0.1&0.0&0.4\\
$x$&0.0&0.1&0.6\\
$y$&0.1&0.2&2.0\\
      $z$&0.2&0.1&1.8\\
      \midrule
      Cost&\$0.07&\$0.05\\
      \bottomrule
    \end{tabular}
  \par}
\end{onlyenv}

% \min Z = 0.07x_1 + 0.05x_2
% 01x_1  \geq 0.4
% 0.1x_2 \geq 0.6
% 0.1x_1 + 0.2x_2 \geq 2.0
% 0.2x_1 + 0.1x_2 \geq 1.8
\end{frameExample}


\begin{frame}{Economic Interpretation}{}

    \only<1>{%
  Mohan-Meakins Breveries Ltd. has two bottling plants, one located at Solan and the other at Mohan Nagar. Each plant produces three drinks, whisky, beer and fruit juices named $A$, $B$, and $C$ respectively. The number of bottles produced per day are as follows:

  {\centering
    \begin{tabular}{rcc}
      \toprule
      &\multicolumn{2}{c}{Plant at}\\
      \cmidrule{2-3}
      &Solan&Mohan Nagar \\
      &$(S)$&$(M)$ \\
      \midrule
      Whisky, $A$&1,500&1,500\\
      Beer, $B$&3,000&1,000\\
      Fruit juices, $C$&2,000&5,000\\
      \bottomrule
    \end{tabular}
    \par}%
  }

  \only<2>{%
  A market survey indicates that during the month of April, there will be a demand of 20,000 bottles of whisky, 40,000 bottles of beer and 44,000 bottles of fruit juices. The operating costs per day for plants at Solan and Mohan Nagar are 600 and 400 monetary units. For how many days each plant be run in April so as to minimize the production cost, while still meeting the market demand?%
  }

  \begin{onlyenv}<3>
      
  \[Z = \min 600x_1 + 400x_2\]

  {\centering
    subject to

    \vspace{3mm}

    \sysdelim..%
    \sysalign{r,r}%
    \systeme[x_1x_2]%
    {
      1\,500x_1 + 1\,500x_2 \geq 20\,000,
      3\,000x_1 + 1\,000x_2 \geq 40\,000,
      2\,000x_1 + 5\,000x_2 \geq 44\,000,
    }
    
    $x_1, x_2 \geq 0$
  \par}
  \end{onlyenv}
  
\end{frame}

%%% Local Variables:
%%% mode: latex
%%% TeX-master: "slides_duality"
%%% End:
