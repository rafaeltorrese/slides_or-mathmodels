
\section{Duality Theorems}
\label{sec:duality-theorems}

\begin{frame}{Teoremas de Dualidad}
  \begin{enumerate} \justifying \parskip4mm
\item<only@1> El dual del dudal es el primal.
\item<only@1> El valor de la función objetivo $Z$ para cualquier solución factible del primal es $\leq$ que el valor de la función objetivo $W$ para cualquier solución factible del dual $Z \leq W$.
\item<only@1> Si el problema primal o dual no tienen solución acotada, entonces, la solución al otro problema (dual o primal) es infactible.
\item<only@1> Si el problema primal y dual tiene soluciones factibles, entonces, la solución para ambos (dual y primal) es la misma, esto es, $\max Z = \min W$
\item<only@1> Holgura complementaria.
\end{enumerate}

\begin{theorem}<only@2>[Holgura Complementaria]
  \begin{enumerate} \justifying \parskip2mm
  \item Si una variable primal es positiva, entonces, la restricción dual correspondiente es una ecuación en el óptimo.
  \item Si la restricción primal es una desigualdad, entonces la variable dual correspondiente es cero en el óptimo.
  \item Si una variable dual es positiva, entonces, la restricción primal correspondiente es una ecuación en el óptimo.
  \item Si una restricción dual es una desigualdad, entonces, la variable primal correspondiente es cero en el óptimo.
  \end{enumerate}
\end{theorem}
\end{frame}

\begin{frame}{Correspondencia entre las Soluciones Primal--Dual}

  \begin{enumerate} \justifying \parskip3mm
  \item Valores de las variables no-básicas del primal se obtienen de la fila base en la solución dual, bajo las variables de holgura (si existen) ignorando el signo negativo y bajo las vraibles artificiales (si no hay variables de holgura en la restricción) ignorando el signo negativo después de borrar la constante $M$.
  \item Los valores de las variables de holgura del problema primal se obtienen de la fila base ($c_j - Z_j$) bajo las variables no-básicas en la solución dual ignorando el signo negativo.
  \item El valor de la función objetivo es el mismo en la solución primal y dual.
  \end{enumerate}
\end{frame}


\begin{frame}{Economic Interpretation}{}
  \[\min 600x_1 + 400x_2\]

  {\centering
    subject to

    \vspace{3mm}

    

  \par}
  
\end{frame}

%%% Local Variables:
%%% mode: latex
%%% TeX-master: "slides_duality"
%%% End:
