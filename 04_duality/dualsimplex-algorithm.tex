\documentclass[letter]{article}

\title{Algoritmo Dual Simplex}


\usepackage[english,spanish]{babel}
\deactivatetilden
\spanishdecimal{.}
\addto\captionsspanish{\def\tablename{Tabla}}
\addto\captionsspanish{\def\listtablename{\'Indice de tablas}}

\usepackage[utf8]{inputenc}
\usepackage{tgbonum}
%\usepackage[T1]{fontenc}
\usepackage{amsmath,amsfonts,amssymb}

\usepackage[utf8]{inputenc}
\usepackage{import} % for \includefrom{path/to/file}{file}
\usepackage{booktabs}
\usepackage{verbatim}
\usepackage{nicefrac}

\usepackage{color} % para color en las letras
\usepackage[table]{xcolor} % para color en las letras

\usepackage{paralist} % para incluir listas dentro de texto (inline lists)


\usepackage%
[
font=small,%
labelfont=bf,%
labelsep=endash,%
tableposition=top,%
skip=3pt%
]{caption} % formato de los captions, puede escribirse solo small,bf, etc
\usepackage{subcaption} % Figures and Tables side by side

\usepackage[%
protrusion=true,%
expansion=true,%
final,%
babel%
]{microtype}

\usepackage{hyperref}

\hypersetup{
  breaklinks=true,%
  colorlinks=true,%
  linkcolor=red,%
  citecolor=black,%
  urlcolor=blue%
}

% For Algorithms
\usepackage{algorithm}
%\usepackage{algorithmicx}
\usepackage{algpseudocode}

\usepackage{systeme}
\renewcommand{\SYSeqnum}{\value{equation}}
\sysautonum{(**)}

\usepackage{graphicx}
\graphicspath{
  {figs/}%
}

\usepackage{tikz} % Tikz must be before ctable package, otherwise
% LaTeX output error 
\usepackage{etoolbox}
\usetikzlibrary{shapes.geometric,arrows,decorations.pathreplacing}

\usepackage{cleveref}

\date{}

\begin{document}

\maketitle
A continuación explicaremos el algoritmo dual simplex. Usaremos el siguiente ejemplo:


\begin{equation}\label{eq:objective}
  \min Z = 2x_1 + 2x_2 + 4x_3
\end{equation}


{\centering
s.a.
\vspace{3mm}

  \sysdelim..%
  \sysalign{r,r}%
  \systeme[x_1x_2x_3]{%
      2x_1 + 3x_2 + 5x_3 \geq 2\gdef\c1{2},
    3x_1 + x_2 + 7x_3  \leq 3,
    x_1 + 4x_2 + 6x_3  \leq 5%
}% systeme

$ x_1, x_2, x_3  \geq 0$ \par}



\section*{Problema en forma canónica}
El primer paso es convertir el problema de programación lineal a la forma canónica de \textbf{un problema de maximización}. En la forma canónica:


\begin{itemize}\parskip3mm
\item los problemas de \textbf{minimización} tienen \textbf{todas} las restricciones en sentido $\geq$ 
\item los problemas de \textbf{maximización} tienen \textbf{todas} las restricciones en sentido $\leq$
\end{itemize}

La función objetivo puede convertirse en un problema de maximización multiplicándola  por $-1$. De esta manera, la función objetivo~(\ref{eq:objective}) se expresa como:


\[  \max G =  -Z = -2x_1 - 2x_2 - 4x_3\]

Para pasar una restricción de $\boldsymbol{\leq}$ a $\boldsymbol{\geq}$ también multiplicamos por $-1$ a toda la restricción. En nuestro ejemplo solo se necesita cambiar el sentido de  la restricción~(\c1) para cumplir con la forma canónica de un problema de maximización. La restricción~(\c1) queda de la siguiente forma \[     -2x_1 - 3x_2 - 5x_3 \leq -2\] 

\section*{Planteamiento del Problema}

El problema de programación lineal de nuestro ejemplo como un problema de maximización se muestra a continuación

\setcounter{equation}{0}

\begin{equation}
  \label{eq:obj2}
  \max G =  -Z = -2x_1 - 2x_2 - 4x_3
\end{equation}

{\centering
  s.t.

\sysdelim..%
\sysalign{r,r}%
\systeme[x_1x_2x_3]{%
    -2x_1 - 3x_2 - 5x_3 \leq -2,
  3x_1 + x_2 + 7x_3  \leq 3,
  x_1 + 4x_2 + 6x_3  \leq 5
}% systeme

\vspace{3mm}
$    x_1, x_2, x_3  \geq 0$ \par}

\section{Formato estándar}
\label{sec:standard}
Después de pasar el problema a su formato canónico de maximización se usa el formato estándar sumando las variables de holgura $S_i$. Nuestro ejemplo ahora se expresa de la siguiente forma.


\setcounter{equation}{0}

\begin{equation}
  \label{eq:obj3}
  \max G = -2x_1 - 2x_2 - 4x_3 + 0S_1  + S_2 + S_3
\end{equation}

{\centering
  s.t.
  \vspace{3mm}
  
\sysdelim..%
\sysalign{r,r}%
\systeme[x_1x_2x_3S_1S_2S_3]{%
    -2x_1 - 3x_2 - 5x_3 + S_1 =  -2,
  3x_1 + x_2 + 7x_3 + S_2  = 3,
  x_1 + 4x_2 + 6x_3 + S_3 = 5%
}% systeme

\vspace{3mm}
$    x_1, x_2, x_3, S_1, S_2, S_3  \geq 0$ \par}

\section*{Tabla inicial dual simplex}
\label{sec:initial-simplex}


A partir del formato estándar la \textbf{tabla dual simplex} inicial se expresa de la siguiente forma



\begin{table}[h]
\caption{Tabla dual simplex inicial.}
      \centering
      \begin{tabular}{rrrrrrrrrr}
        \toprule
        $\max$&$c_j$&-2&-2&-4&0&0&0&\\
        \midrule
        $C_b$&Basis&$x_1$&$x_2$&$x_3$&$s_1$&$s_2$&$s_3$&$b$&\\
        \midrule
        0&\color{blue}$s_1$&-2&-3&-5&1&0&0&-2& \\
        0&$s_2$&3&1&7&0&1&0&3&\\
        0&$s_3$&1&4&6&0&0&1&5&\\
        \midrule
              &$Z_j$&0&0&0&0&0&0&\cellcolor{yellow}\textbf{0}&\\
              &$c_j - Z_j$&-2&-2&-4&0&0&0&\\
        \bottomrule
      \end{tabular}
    \end{table}
    


    \begin{description}
    \item[Paso 1.] Se calculan los elementos $c_j - Z_j$ para cada columns. Después, pueden surgir tres casos:
  \begin{enumerate}  
  \item  Si todos los elementos $c_j - Z_j$ son negativos o zero y todos los elementos $b_i$ (lado derecho) son no-negativos, la solución es factible y óptima.
\item Si todos los elementos $c_j - Z_j$ son negativos o zero y al menos un elemento de $b_i$ es negativo, se procede al paso número 2
\item  Si algún elemento de $c_j - Z_j$ es positivo, el algoritmo no funciona.
  \end{enumerate}
\item[Paso 2.] Selecciona la fila que contiene los $b_i$ más negativos. Esta fila se llama fila clave o fila pivote. La variable básica correspondiente sale de la base. Esto se llama condición de factibilidad dual
\item[Paso 3.]  Mira los elementos de la fila clave.
  \begin{enumerate}[a)]
  \item  Si todos los elementos son no negativos, el problema no tiene solución factible.
  \item  Si al menos un elemento es negativo, encuentre los ratios de los elementos correspondientes de la fila a estos elementos. \textbf{Ignorar ratios asociados con elementos positivos o cero de la fila clave}. Elige la menor de estas proporciones. La columna correspondiente es la columna clave y la variable asociada es la variable de entrada. A esto se le llama condición de óptimo dual. Marque el elemento clave o el elemento pivote.
  \end{enumerate}
\end{description}

\begin{table}[h]
\caption{Identificar variable de salida.}
      \centering
      \begin{tabular}{rrrrrrrrrr}
        \toprule
        $\max$&$c_j$&-2&-2&-4&0&0&0&\\
        \midrule
        $C_b$&Basis&$x_1$&$x_2$&$x_3$&$s_1$&$s_2$&$s_3$&$b$&\\
        \midrule
        0&\color{blue}$s_1$&-2&-3&-5&1&0&0&\cellcolor{blue!30}-2&\textrightarrow \\
        0&$s_2$&3&1&7&0&1&0&3&\\
        0&$s_3$&1&4&6&0&0&1&5&\\
        \midrule
              &$Z_j$&0&0&0&0&0&0&\cellcolor{yellow}\textbf{0}&\\
              %&$c_j - Z_j$&-2&-2&-4&0&0&0&\\
        \bottomrule
      \end{tabular}
    \end{table}
    
\begin{table}[h]
\caption{Fila pivote.}
    \centering
    \begin{tabular}{rrrrrrrrrr}
      \toprule
      $\max$&$c_j$&-2&-2&-4&0&0&0&\\
      \midrule
      $C_b$&Basis&$x_1$&$\cellcolor{yellow}x_2$&$x_3$&$s_1$&$s_2$&$s_3$&$\pmb{b}$&\\
      \midrule
      0&$s_1$&\cellcolor{blue!30}-2&\cellcolor{blue!30}-3&\cellcolor{blue!30}-5&\cellcolor{blue!30}1&\cellcolor{blue!30}0&\cellcolor{blue!30}0&-2& key row\\
      0&$s_2$&3&1&7&0&1&0&3&\\
      0&$s_3$&1&4&6&0&0&1&5&\\
      \midrule
      &$Z_j$&0&0&0&0&0&0&\textbf{0}&\\
      &$c_j - Z_j$&\cellcolor{blue!30}-2&\cellcolor{blue!30}-2&\cellcolor{blue!30}-4&\cellcolor{blue!30}0&\cellcolor{blue!30}0&\cellcolor{blue!30}0&\\
      \bottomrule
    \end{tabular}
    \end{table}

\section*{Iteraciones}
\label{iterations}

    
    \begin{description}
  \item[Step 4.] Calcule $c_j - Z_j$ donde $Z_j = \sum c_B a_{ij}$. Como todos $c_j - Z_j$ son negativos o cero y $b_1$ es negativo, la solución es óptima pero no factible.
  \item[Step 5.] Como $b_1 = -2$, la primera fila es la fila clave y $s_1$ es la variable saliente.
  \end{description}

      \begin{table}[h]
    \caption{Calculo de los ratios solo con valores negativos en fila pivote.}
        \centering
    \begin{tabular}{rrrrrrrrrr}
      \toprule
      $\max$&$c_j$&-2&-2&-4&0&0&0&\\
      \midrule
      $C_b$&Basis&$x_1$&$\cellcolor{yellow}x_2$&$x_3$&$s_1$&$s_2$&$s_3$&$\pmb{b}$&\\
      \midrule
      0&$s_1$&\cellcolor{blue!30}-2&\cellcolor{blue!30}-3&\cellcolor{blue!30}-5&1&0&0&-2& solo negativos\\
      0&$s_2$&3&1&7&0&1&0&3&\\
      0&$s_3$&1&4&6&0&0&1&5&\\
      \midrule
      &$Z_j$&0&0&0&0&0&0&\textbf{0}&\\
      &$c_j - Z_j$&\cellcolor{blue!30}-2&\cellcolor{blue!30}-2&\cellcolor{blue!30}-4&0&0&0&\\
      \bottomrule
    \end{tabular}
  \end{table}
  %%%
  
  \begin{description}
  \item[Step 6.] Encuentre los elementos de los ratios de la fila $c_j - Z_j$, esto es dividir los elemeentos del vector $c_j - Z_j$ entre los elementos de la fila correspondiente a la variable de salida. Ignore las proporciones correspondientes a elementos positivos o cero para la fila clave. Las proporciones deseadas son: \[ \frac{-2}{-2} = 1 \quad\quad \frac{-2}{-3}=\frac{\color{blue}2}{\color{blue} 3} \quad\quad \frac{-4}{-5} = \frac{4}{5} \] Dado que $\frac{2}{3}$ es la proporción más pequeña, la columna $x_2$ es la columna clave ; $x_2$ es la variable entrante y $-3$ es el elemento clave.

      \begin{table}[h]
    \caption{Calculo de los ratios.}
        \centering
    \begin{tabular}{rrrrrrrrrr}
      \toprule
      $\max$&$c_j$&-2&-2&-4&0&0&0&\\
      \midrule
      $C_b$&Basis&$x_1$&$\cellcolor{yellow}x_2$&$x_3$&$s_1$&$s_2$&$s_3$&$\pmb{b}$&\\
      \midrule
      0&$s_1$&\cellcolor{blue!30}-2&\cellcolor{blue!30}-3&\cellcolor{blue!30}-5&1&0&0&-2& solo negativos\\
      0&$s_2$&3&1&7&0&1&0&3&\\
      0&$s_3$&1&4&6&0&0&1&5&\\
      \midrule
      &$Z_j$&0&0&0&0&0&0&\textbf{0}&\\
            &$c_j - Z_j$&\cellcolor{blue!30}-2&\cellcolor{blue!30}-2&\cellcolor{blue!30}-4&0&0&0&\\
      \midrule
      &ratios&\cellcolor{green!30}1&\cellcolor{green!30}\nicefrac{2}{3}&\cellcolor{green!30}\nicefrac{4}{5}&0&0&0&\\
      \bottomrule
    \end{tabular}
  \end{table}
    
  \item[Step 7.] Reemplazar $s_1$ por $x_2$
  \end{description}

  
    \begin{table}[h]
      \caption{Localización del elemento pivote}
  \centering     
    \begin{tabular}{rrrrrrrrrr}
      \toprule
      $\max$&$c_j$&-2&-2&-4&0&0&0&\\
      \midrule
      $c_B$&Basis&$x_1$&$x_2$&$x_3$&$s_1$&$s_2$&$s_3$&$b$&\\
      \midrule
      -2&$x_2$&-2&\cellcolor{orange}-3&-5&1&0&0&-2&\\
      0&$s_2$&3&1&7&0&1&0&3&\\
      0&$s_3$&1&4&6&0&0&1&5&\\
      \midrule
      &$Z_j$&0&0&0&0&0&0&\textbf{0}&\\
            &$c_j - Z_j$&-2&-2&-4&0&0&0&\\
            &&&\textuparrow&&&&&\\
      \bottomrule
    \end{tabular}
    \end{table}
    

    Aplicando Gauss-Jordan tenemos la primer iteración
    

    \begin{table}[h]
      \caption{Primera Iteración aplicando algoritmo Dual Simplex}
      \centering
    \begin{tabular}{rrrrrrrrrr}
      \toprule
      $\max$&$c_j$&-2&-2&-4&0&0&0&\\
      \midrule
      $c_B$&Basis&$x_1$&$x_2$&$x_3$&$s_1$&$s_2$&$s_3$&$b$&\\
      \midrule
      -2&$x_2$&\nicefrac{2}{3}&1&\nicefrac{5}{3}&-\nicefrac{1}{3}&0&0&\nicefrac{2}{3}&\\
      0&$s_2$&\nicefrac{7}{3}&0&\nicefrac{16}{3}&\nicefrac{1}{3}&1&0&\nicefrac{7}{3}&\\
      0&$s_3$&-\nicefrac{5}{3}&0&-\nicefrac{2}{3}&\nicefrac{4}{3}&0&1&\nicefrac{7}{3}&\\
      \midrule
      &$Z_j$&-\nicefrac{4}{3}&-2&-\nicefrac{10}{3}&\nicefrac{2}{3}&0&0&\cellcolor{yellow}-\nicefrac{4}{3}&\\
            &$c_j - Z_j$&-\nicefrac{2}{3}&0&-\nicefrac{2}{3}&-\nicefrac{2}{3}&0&0&\\
            %&&&\multicolumn{6}{r}{Solución básica factible óptima}\\
      \bottomrule
    \end{tabular}
    \end{table}



        \begin{table}[h]
      \caption{Solución aplicando algoritmo Dual Simplex}
      \centering
    \begin{tabular}{rrrrrrrrrr}
      \toprule
      $\max$&$c_j$&-2&-2&-4&0&0&0&\\
      \midrule
      $c_B$&Basis&$x_1$&$x_2$&$x_3$&$s_1$&$s_2$&$s_3$&$b$&\\
      \midrule
      -2&$x_2$&\nicefrac{2}{3}&1&\nicefrac{5}{3}&-\nicefrac{1}{3}&0&0&\cellcolor{cyan}\nicefrac{2}{3}&\\
      0&$s_2$&\nicefrac{7}{3}&0&\nicefrac{16}{3}&\nicefrac{1}{3}&1&0&\cellcolor{cyan}\nicefrac{7}{3}&\\
      0&$s_3$&-\nicefrac{5}{3}&0&-\nicefrac{2}{3}&\nicefrac{4}{3}&0&1&\cellcolor{cyan}\nicefrac{7}{3}&\\
      \midrule
      &$Z_j$&-\nicefrac{4}{3}&-2&-\nicefrac{10}{3}&\nicefrac{2}{3}&0&0&\cellcolor{yellow}-\nicefrac{4}{3}&\\
            &$c_j - Z_j$&\cellcolor{cyan}-\nicefrac{2}{3}&\cellcolor{cyan}0&\cellcolor{cyan}-\nicefrac{2}{3}&\cellcolor{cyan}-\nicefrac{2}{3}&\cellcolor{cyan}0&\cellcolor{cyan}0&\\
            &&&\multicolumn{6}{r}{Solución básica factible óptima}\\
      \bottomrule
    \end{tabular}
  \end{table}
  
    \section*{Solución óptima.}

    Como todos los $c_j - Z_j$ son negativos o cero y todos los $b_i$ son positivos, la solución dada por la tabla es óptima. La solución óptima es:    
    

    \begin{table}[h]
      \caption{Solución al problema}
\centering
   \begin{tabular}{cccc}
     $x_1$&$x_2$&$x_3$& $\min Z$\\
     \midrule
     0&$\nicefrac{2}{3}$&0& $\cellcolor{yellow}\nicefrac{4}{3}$
   \end{tabular}
 \end{table}
 
    

\end{document}