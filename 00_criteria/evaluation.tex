\section{Criterios de evaluación}
\begin{frame}{Criterios de Evaluación}
  
  \begin{table}[h]
    \centering
    \begin{tabular}{lcl}
      \toprule[2pt]
      \textbf{Actividad} & \textbf{Ponderación} & \multicolumn{1}{c}{\textbf{Fecha}}\\ \midrule
      Homework & 20 & \\
      Exam1 & 10 & Feb 3th  \\
      MidTerm & 20 & Mar 3th   \\
      Exam3 & 10 & Apr 14th   \\
      Final Exam & 40 & May 13th   \\
      \midrule
      TOTAL & 100 \\ \bottomrule[1.5pt]
    \end{tabular}
    \caption{Criterios de evaluación.}
    \label{tab:criteriosEvaluacion}
  \end{table}
\end{frame}


\begin{comment}
  
\begin{frame}
  \frametitle{Lineamientos}
  \begin{itemize} \justifying \parskip3mm

  \item<only@1> 15 a 30 minutos retardo, después de este tiempo se considera falta.
      \item<only@1> Si llegan tarde, por favor entrar en silencio.
      \item<only@1> Retardos, se colocan al final de la clase.
      \item<only@1> Respeto, actitud, evitar groserías.
      \item<only@1> Copiar exámenes o tareas anula calificación  correspondiente.
      \item<only@2> Examenes se aplican \alert{solo en fecha y hora establecida}.
      \item<only@2>  Pueden  \alert{tomar alimentos muy ligeros y sin olores fuertes} dentro del salón de clase,
      \item<only@2> Encender equipos al llegar
  \end{itemize}
\end{frame}
\end{comment}





%%% Local Variables:
%%% mode: latex
%%% TeX-master: "slides"
%%% End:
