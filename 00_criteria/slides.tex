\documentclass{beamer}

\usepackage[spanish]{babel}
\decimalpoint
\usepackage[utf8]{inputenc}
\usepackage{ragged2e} % este paquete es para justificar.
\justifying
\usepackage{booktabs}
\usepackage{verbatim}
\usepackage{amsmath}
\usepackage{bm} % bold font greek letters. Use \boldsymbol{ }  and \pmb
\usepackage{xfrac} % for slanted fractions   \sfrac{}{}
\usepackage{nicefrac} % for small fractions in text or math mode   \nicefrac{}{}
%\usepackage{blkarray} % for block arrays, useful for markov chains. Error with Metropolis Beamer Theme
\usepackage{hhline} % hlines  but interacts with vertical lines
\usepackage{multirow}
\usepackage{pgfpages} % for handouts
%
\usepackage{import} 
%
%\pgfpagesuselayout{4 on 1}[letterpaper,landscape]
\usepackage[font=scriptsize,labelfont=bf,justification=centerlast,format=hang]{caption} %To change the appearance of captions
% tikz
\usepackage{tikz} % for draw
\usetikzlibrary{automata,arrows,positioning,calc}
\graphicspath{{figs/}}


\definecolor{tealsection}{RGB}{50,90,90}
\hypersetup{colorlinks=true, urlcolor=red , linkcolor=tealsection}



% ========================================
\newenvironment<>{exercise}[1]
{
\setbeamercolor{block title}{fg=white,bg=orange!45!black}
\begin{block}#2{Ejercicio #1}\justifying
}
{%
\end{block}
}%
% ========================================
\newenvironment<>{solution}[1]
{
\setbeamercolor{block title}{fg=white,bg=orange!65!black}
\begin{block}#2{Ejercicio #1 (Solución)}\justifying
}
{
\end{block}
}
% ========================================
\newenvironment<>{example}[1]
{
\setbeamercolor{block title}{fg=white,bg=green!65!black}
\begin{block}#2{Ejemplo #1}\justifying
}
% content
{
\end{block}
}
% ========================================
\newenvironment<>{frameExample}[2]
{
\setbeamercolor{frametitle}{fg=white,bg=green!60!blue}
\begin{frame} \justifying
  \frametitle{Ejemplo #1}
  \framesubtitle{ #2}
  
}
% content
{
\end{frame}
}
% ========================================
\newenvironment<>{framebb}[1] % for blackboard exercises
{
%\usebackgroundtemplate{\includegraphics[scale=0.3]{blackboard-logo.png}}
\setbeamercolor{frametitle}{fg=white,bg=black!85}
\begin{frame} \justifying
  \frametitle{%
  \includegraphics[scale=0.08]{blackboard-logo.png} \hspace{5mm} Contenidos y Materiales/Homework/hw\textunderscore #1
  }
%{\centering \includegraphics[scale=0.5]{BlackBoardSymbol.png}\par}
%\vspace{5mm}
Hacer actividad en Parejas. Cada alumno subirá la \alert{actividad a su cuenta individual} de Blackboard \\[5mm]
}
{
\end{frame}
}
% ========================================
\newenvironment<>{framemtb} % for Using Minitab
{
%\usebackgroundtemplate{\includegraphics[scale=0.3]{blackboard-logo.png}}
\setbeamercolor{frametitle}{fg=white,bg=green!60!black}
\begin{frame}\justifying
\frametitle{\includegraphics[scale=0.1]{minitab-logo.png} \hspace{4mm} Minitab}
%{\centering \includegraphics[scale=0.5]{BlackBoardSymbol.png}\par}
%\vspace{5mm}

}
{
\end{frame}
}
% ========================================
%\graphicspath{}
\usetheme[
sectionpage=progressbar,
numbering=none,
progressbar=frametitle
]{metropolis}

% \usetheme{Boadilla}

\title{Investigación De Operaciones \\ Modelos Matemáticos}
\subtitle{Criterios de Evaluación} 


\institute{UNIVERSIDAD ANÁHUAC MÉXICO}
\author{Rafael Torres Escobar, Ph.D.}
\date[Anáhuac México]{}


 \AtBeginSection[] % Do nothing for \section* %
{
\begin{frame}<beamer> 
  \frametitle{Agenda}
  \tableofcontents[currentsection] 
\end{frame}
}


%%%%%%%%%%%%%%%%%%%%%%%%%%%%%%

\begin{document}

\begin{frame}
  \titlepage
  Hola
\end{frame}


     \begin{frame}{Agenda}
   \tableofcontents
 \end{frame}



 %\section{Presentación Del Curso}
\begin{frame}{Objetivos}

  \begin{itemize} \justifying \parskip3mm

\item	Analiza los métodos de optimización lineal utilizados en la investigación de operaciones (IO), para implementarlos en la organización.
\item	Distingue los diferentes modelos matemáticos que apoyan la toma de decisiones en las organizaciones.
\item	Construye modelos de optimización para la resolución de problemas de distinta índole en diferentes industrias utilizando herramientas de cómputo  



  \end{itemize}

  
  
\end{frame}

\begin{frame}
  \frametitle{Temas del Curso}
  \begin{enumerate} \justifying \parskip3mm
  \item<only@1> Modelos matemáticos.
\item<only@1> Programación lineal
\item<only@1> Programación lineal entera.
\item<only@1> Modelos de Red.
\item<only@1> Programación dinámica.


  \end{enumerate}

\end{frame}




%%% Local Variables:
%%% mode: latex
%%% TeX-master: "slides"
%%% End:

 \section{Criterios de evaluación}
\begin{frame}{Criterios de Evaluación}
  
  \begin{table}[h]
    \centering
    \begin{tabular}{lcr}
      \toprule[2pt]
      \textbf{Actividad} & \textbf{Ponderación} & \multicolumn{1}{c}{\textbf{Fecha}}\\ \midrule
      Homework & 20 & \\
      Exam1 & 10 & Feb 03th  \\
      MidTerm & 20 & Mar 03th   \\
      Exam3 & 10 & Apr 14th   \\
      Final Exam & 40 & May 18th   \\
      \midrule
      TOTAL & 100 \\ \bottomrule[1.5pt]
    \end{tabular}
    \caption{Criterios de evaluación.}
    \label{tab:criteriosEvaluacion}
  \end{table}
\end{frame}


\begin{comment}
  
\begin{frame}
  \frametitle{Lineamientos}
  \begin{itemize} \justifying \parskip3mm

  \item<only@1> 15 a 30 minutos retardo, después de este tiempo se considera falta.
      \item<only@1> Si llegan tarde, por favor entrar en silencio.
      \item<only@1> Retardos, se colocan al final de la clase.
      \item<only@1> Respeto, actitud, evitar groserías.
      \item<only@1> Copiar exámenes o tareas anula calificación  correspondiente.
      \item<only@2> Examenes se aplican \alert{solo en fecha y hora establecida}.
      \item<only@2>  Pueden  \alert{tomar alimentos muy ligeros y sin olores fuertes} dentro del salón de clase,
      \item<only@2> Encender equipos al llegar
  \end{itemize}
\end{frame}
\end{comment}





%%% Local Variables:
%%% mode: latex
%%% TeX-master: "slides"
%%% End:




\end{document}
