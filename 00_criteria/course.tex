\section{Presentación Del Curso}
\begin{frame}{Objetivos}

  \begin{itemize} \justifying \parskip3mm

\item	Analiza los métodos de optimización lineal utilizados en la investigación de operaciones (IO), para implementarlos en la organización.
\item	Distingue los diferentes modelos matemáticos que apoyan la toma de decisiones en las organizaciones.
\item	Construye modelos de optimización para la resolución de problemas de distinta índole en diferentes industrias utilizando herramientas de cómputo  



  \end{itemize}

  
  
\end{frame}

\begin{frame}
  \frametitle{Temas del Curso}
  \begin{enumerate} \justifying \parskip3mm
  \item<only@1> Modelos matemáticos.
\item<only@1> Programación lineal
\item<only@1> Programación lineal entera.
\item<only@1> Modelos de Red.
\item<only@1> Programación dinámica.


  \end{enumerate}

\end{frame}




%%% Local Variables:
%%% mode: latex
%%% TeX-master: "slides"
%%% End:
