\begin{frameExample}{Product Mix Problem}{}
  % Example 2.6-16 Gupta Ebook
  \begin{onlyenv}<1>
      A manufacturer of biscuits is considering four ytpes of gift-packs containing three types of biscuits: orange cream(o.c.) chocolate cream (c.c.) and wafers (w.). Market research conducted to assess the preferences of the costumers shows the following types of assortments to be in good demand:

  {
    \centering
    \begin{tabular}{clc}
      \toprule
      Assortment    & Contents&	Selling price/kg \$\\
      \midrule
A&	Not less than 40 \% of o.c.&	200\\
&	Not more than 20\% of c.c.&	\\
B&	Not less than 20\% of o.c.&	250\\
&	Not more than 40\% of c.c.&	\\
C&	Not less than 50\% of o.c.&	220\\
&	Not more than 10\% of c.c.&	\\
      D&	No restrictions	&120\\
      \bottomrule
    \end{tabular}
    \par
  }
\end{onlyenv}

\begin{onlyenv}<2>
  For the biscuits the manufacturing capacity and costs are given below.
  
  {
    \centering
    \begin{tabular}{ccc}
      \toprule
      Biscuit variety&	Plant capacity&	Manufacturing cost\\
                     &(kg / day)&	(\$ / kg)\\
      \midrule
      o.c.&	200&	80\\
      c.c.	&200&	90\\
      w.	&150&	70\\
      \toprule
    \end{tabular}
    \par
  }

  Formulate the L.P. model to finde the production schedule to find the production schedule which maximizes the profit assuming that there are no market restrictions.
\end{onlyenv}
\end{frameExample}


%%% Local Variables:
%%% mode: latex
%%% TeX-master: "../slides_linear-programming-intro"
%%% End:
