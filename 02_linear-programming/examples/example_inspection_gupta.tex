\begin{frameExample}{Inspección}{}
  % EXAMPLE 2.6-5 {lnspection Problem} Gupta
Una empresa tiene dos grados de inspectores, I y II para llevar a cabo la inspección de control de calidad. Se deben inspeccionar al menos 1,500 piezas en un día de 8 horas. El inspector de grado I puede \alert{verificar 20 piezas en una hora} con una precisión del 96\%. El inspector grado II  \alert{verifica 14 piezas por hora} con un precisión del 92\%. Los salarios del inspector de grado I son \$ 5 por hora, mientras que los del inspector grado II  son \$ 4 por hora. Cualquier \alert{error} cometido por un inspector \alert{cuesta \$ 3 a la empresa}. Si hay, en total, 10 inspectores grado I
 y 15 inspectores de grado II en la empresa, encuentre la asignación óptima de inspectores que minimiza el costo diario de inspección.
    
\end{frameExample}



%%% Local Variables:
%%% mode: latex
%%% TeX-master: "../slides"
%%% End:
