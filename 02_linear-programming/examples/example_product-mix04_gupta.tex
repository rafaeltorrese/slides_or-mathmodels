\begin{frameExample}{Product Mix Problem}{}
  \begin{onlyenv}<1>
    A certain farming organization operates three farms of comparable productivity. The output of each farm is limited both by the usable acreage and by the amount of water available for irrigation. Following are the data for the upcoming season

  {\centering
    \scalebox{0.8}{%
\begin{tabular}{crr}
      \toprule
      Farm&\multicolumn{1}{c}{Usable acreage}&\multicolumn{1}{c}{Water available} \\
          &&\multicolumn{1}{c}{in acre feet} \\
      \midrule
          1&400&1,500 \\
          2&600&2,000 \\
      3&300&900\\
      \bottomrule
    \end{tabular}
    }
  \par}
  The organization is considering three crops for planting which differ primarily in their expected profit per acre and in their consumption of water. Furthermore, the total acreage that can be devoted to each of the crops is limited by the amount of appropriate harvesting equipment available.
\end{onlyenv}

\begin{onlyenv}<2>
  {\centering
    \scalebox{0.8}{%
      \begin{tabular}{cccc}
        \toprule
        Crop & Minimum acreage& Water consumption& Expected profit\\
             & &in acre feet per acre& per acre \$\\
        \midrule
        A&400&5&400\\
        B&300&4&300\\
        C&300&3&100\\
        \bottomrule
      \end{tabular}
    }
    \par}

  In order to maintain a uniform work load among the farms, it is the policy of the organization that the percentage of the usable acreage planted must be the same at each farm. However, any combination of the crops may be grown at any of the farms. The organization wishes to know how much of each crop should be planted at the respective farms in orders to maximize expected profit. Formulate this as a linear programming problem.
\end{onlyenv}
\end{frameExample}


%%% Local Variables:
%%% mode: latex
%%% TeX-master: "../slides_linear-programming-intro"
%%% End:
