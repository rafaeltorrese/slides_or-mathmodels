\begin{frameExample}{Producción}{}
  % EXAMPLE 2.6-1 (Production Allocation Problem} Gupta ebook
  Una empresa produce tres productos. Estos productos se procesan en tres máquinas diferentes. El tiempo requerido para fabricar una unidad de cada uno de los tres productos y la capacidad diaria de las tres máquinas se detallan en la tabla a continuación.

  {\centering
    \scalebox{0.8}{
      \begin{tabular}{ccccc}
      \toprule
        &\multicolumn{3}{c}{Time per unit (minutes)}&Machine capacity\\
        \cmidrule{2-5}
      Machine&Product 1&Product 2&Product 3&(minutes /day)\\
      \midrule
      $M_1$&2&3&2&440\\
      $M_2$&4&--&3&470\\
      $M_3$&2&5&--&430\\
      \bottomrule
    \end{tabular}
    } % \scalebox
\par}

   Se requiere determinar la cantidad diaria de unidades que se fabricarán para cada producto. El beneficio por unidad para el producto 1, 2 y 3 es de \$ 4, \$ 3 y \$ 6 respectivamente. Se supone que todas las cantidades producidas se consumen en el mercado. Formule el modelo matemático (L.P.) que maximizará la ganancia diaria.
\end{frameExample}



%%% Local Variables:
%%% mode: latex%%% TeX-master: "slides_linear-programming-intro"
%%% End:
