\begin{frameExample}{Mezcla de Productos}{}
  % EXAMPLE 2.6-6 (Product Mix Problem) Gupta
Una compañía química produce dos productos, $X$ e $Y$. Cada unidad de producto $X$ requiere 3 horas en operación I y 4 horas en operación II, mientras que cada unidad de producto $Y$ requiere 4 horas en operación I y 5 horas en operación II. El tiempo total disponible para las operaciones I y II es 20 horas y 26 horas respectivamente. La producción de cada unidad de producto $Y$ también da como resultado dos unidades de un subproducto $Z$ sin costo adicional. El producto $X$ se vende con una ganancia de \$ 10 / unidad, mientras que $Y$ se vende con una ganancia de \$ 20 / unidad. El subproducto $Z$ aporta un beneficio unitario de \$ 6 si se vende; en caso de que no se pueda vender, el costo de destrucción es de \$ 4 / unidad. Los pronósticos indican que no se pueden vender más de 5 unidades de $Z$. Formule el modelo L.P. para determinar las cantidades de $X$ e $Y$ que se producirán, teniendo en cuenta $Z$, de modo que la ganancia obtenida sea máxima.
    
\end{frameExample}

\begin{frameExample}{Mezcla de Productos}{}
  Sea $x_1 , x_2, x_z$ el número de productos $X$, $Y$, $Z$ que se van a producir, tenemos que:
  \begin{align*}
    x_z & = \text{número de productos tipo } Z \\
        & = \text{número de unidades vendidas de } Z + \text{unidades destruidas }Z\\
          &= x_3 + x_4
  \end{align*}
\end{frameExample}

\begin{frameExample}{Mezcla de Productos}{}
  \begin{flalign*}
    \max Z = 10x_1 + 20x_2 + 6x_3 - 4x_4 & \\
    3x_1 + 4x_2 & \leq 20\\
    4x_1 + 5x_2 & \leq 26\\
    x_3 & \leq 5\\[3mm]
    2Y & = Z\\
    2x_2 & = x_3 + x_4\\[5mm]
    x_1, x_2, x_3, x_4 & \geq 0
  \end{flalign*}
\end{frameExample}
%%% Local Variables:
%%% mode: latex
%%% TeX-master: "../slides"
%%% End:
