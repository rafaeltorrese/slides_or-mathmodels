\begin{frameExample}{Production Planning}{}
  % Example 2.6-11 Gupta ebook
  \only<1>{%
    Una empresa que fabrica enfriadores de aire, en la actualidad, tiene pedidos  para los próximos 6 meses. La empresa puede programar su producción durante los próximos 6 meses para cumplir con los pedidos de forma regular o en horas extra. El tamaño del pedido y los costos de producción durante los próximos seis meses son los siguientes:%
  }

  
  \begin{onlyenv}<1,2>
    {%
    \centering
    \scalebox{0.9}{%
      \begin{tabular}{lrrrrrrr}
        \toprule
        &&\multicolumn{6}{c}{Month}\\
        \cmidrule{3-8}
      &&1&2&3&4&5&6\\
      \midrule
      Orders&:&640&660&700&750&550&650\\
      Cost/unit(\$) for&& & & & & & \\
      regular production&:&40&42&41&45&39&40\\
      Cost/unit(\$) for& & & & & & & \\
      overtimeproduction&:&52&50&53&50&45&43\\
      \bottomrule
    \end{tabular}
    }% scalebox
    \par
  }
  \end{onlyenv}
  
  

  \only<2>{%
    Con 100 enfriadores de aire en existencia en la actualidad, la empresa desea tener al menos 150 enfriadores de aire en existencia al final de los 6 meses. La producción regular y extraordinaria en cada mes no debe exceder las 600 y 400 unidades respectivamente. El costo de manejo de inventario para enfriadores de aire es de \$12 por unidad por mes. Formule el modelo de programación lineal (P.L.) para minimizar el costo total.%
  }

  \only<3>{%
    La decisión consiste en determinar el número de unidades de enfriadores que se van a producir en tiempo regular y tiempo extra además de considerar el número de unidades en inventario al final de cada mes.

    Sea $x_{ij}$ el número de unidades fabricadas en el mes $j\, (j = 1, 2, \ldots, 6)$ en tiempo regular o extra $i, (i = 1, 2)$. Además sea $y_j$ el número de unidades en inventario al finalizar el mes $j\, (j = 1, 2, \ldots, 6)$.

    El objetivo es minimizar el costo total (\alert{producir y tener en inventario})
    \begin{flalign*}
      \min Z  =\;\; & 40x_{11} + 42x_{12} + 41x_{13} + 45x_{14} + 39x_{15} + 40x_{16} + \\
      & 52x_{21} + 50x_{22} + 53x_{23} + 50x_{24} + 45x_{25} + 43x_{26} + \\
      & 12(y_{1} + y_{2} + y_{3} + y_{4} + y_{5} + y_{6} )\\
    \end{flalign*}
    % 
    
  }

  \only<4>{%
    Las restricciones son:

    {
      \centering
      \begin{tabular}{rll}
        para el primer mes & $100 + x_{11} + x_{21} - 640$ & $= y_1$\\
        para el segundo mes & $y_1 + x_{12} + x_{22} - 660$ & $= y_2$\\
        para el tercer mes & $y_2 + x_{13} + x_{23} - 700$ & $= y_3$\\
        para el cuarto mes & $y_3 + x_{14} + x_{24} - 750$ & $= y_4$\\
        para el quinto mes & $y_4 + x_{15} + x_{25} - 550$ & $= y_5$\\
        para el sexto mes & $y_5 + x_{16} + x_{26} - 650$ & $= y_6$\\        
      \end{tabular}
      \par}
  }
  \only<5>{%
    \begin{columns}[t]
      \column{0.2\textwidth}
      Restricción de tiempo regular
\begin{align*}
      x_{11} &\leq 600\\
      x_{12} &\leq 600\\
      \vdots &\leq 600\\
      \vdots &\leq 600\\
      x_{16} &\leq 600\\       
    \end{align*}
    \column{0.2\textwidth}
    Restricción de tiempo extra
\begin{align*}
      x_{21} &\leq 400\\
      x_{22} &\leq 400\\
      \vdots &\leq 400\\
      \vdots &\leq 400\\
      x_{26} &\leq 400\\       
\end{align*}
\column{0.2\textwidth}
Condición de no-negatividad
\begin{align*}
  x_{11} &\geq 0\\
  x_{12} &\geq 0\\
  \vdots &\geq 0\\
  \vdots &\geq 0\\
  x_{16} &\geq 0\\
  \end{align*}
  \column{0.2\textwidth}
  Condición de no-negatividad
\begin{align*}
  x_{21} &\geq 0\\
  x_{22} &\geq 0\\
  \vdots &\geq 0\\
  \vdots &\geq 0\\
  x_{26} &\geq 0\\
\end{align*}
\column{0.2\textwidth}
Condición de no-negatividad
\begin{align*}
  y_{1} &\geq 0\\
  y_{2} &\geq 0\\
  \vdots &\geq 0\\
  \vdots &\geq 0\\
  y_{6} &\geq 0\\
  \end{align*}
    \end{columns}    
  }% end only
\end{frameExample}



%%% Local Variables:
%%% mode: latex
%%% TeX-master: "../slides"
%%% End:
