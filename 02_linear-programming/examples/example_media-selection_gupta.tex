\begin{frameExample}{Seleccion de Medios}{}
  % LE 2.6-4 (Advertising Media Selection Problem) 

  \only<1>{%
  Una empresa de publicidad desea planificar su estrategia publicitaria en tres medios diferentes de televisión, radio y revistas. El objetivo de la publicidad es llegar al mayor número posible de clientes posibles. Se han obtenido los siguientes datos de una encuesta de mercado:%
  }

  {\centering

    \scalebox{0.7}{
      \begin{tabular}{rrrrr}
        \toprule
        &Television&Radio&Magazine I&Magazine II \\
        \toprule
        Cost of and advertising unit&30,000&20,000&15,000&10,000 \\[3mm]
        No. of potential customers&&&& \\
        reached per unit&200,000&600,000&150,000&100,000 \\[3mm]
        No. of female customers&&&& \\
        reached per unit&150,000&400,000&70,000&50,000\\
        \bottomrule
      \end{tabular}
    }     
%\includegraphics<1,2>[scale=0.5]{example_selection-media_gupta}
\par}

\only<2>{%
La compañía quiere gastar no más de \$ 450,000 en publicidad. Los siguientes son los requisitos adicionales que deben cumplirse:
\begin{enumerate}[i)] \justifying
\item  se producen al menos 1 millón de exposiciones entre clientes femeninas,
\item  la publicidad en revistas se limitará a \$ 150,000
\item  se deben comprar al menos 3 unidades publicitarias en la revista I y 2 unidades en la revista II
\item  el número de unidades publicitarias en televisión y radio debe ser entre 5 y 10 cada uno.
\end{enumerate}

Formular un modelo L.P. para el problema.%
}
    
\end{frameExample}



%%% Local Variables:
%%% mode: latex
%%% TeX-master: "slides_linear-programming-intro"
%%% End:
