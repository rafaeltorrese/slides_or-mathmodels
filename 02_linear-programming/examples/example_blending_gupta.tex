\begin{frameExample}{Mezcla}{}
  % EXAMPLE 2.6-3 (Blending Problem) 
  \only<1>{%
    Una empresa produce una aleación que tiene las siguientes especificaciones:

\begin{enumerate}[i)]  \justifying
\item  gravedad específica $\leq$ 0.98,
\item  cromo $\geq$ 8\%,
\item  punto de fusión $\geq$ 450 °C.
\end{enumerate}

Las materias primas A, B y C que tienen las propiedades que se muestran en la tabla pueden usarse para hacer la aleación.%
}

{\centering

  \scalebox{0.8}{%
\begin{tabular}{rrrr}
    \toprule
    &\multicolumn{3}{c}{Properties of}\\
    &\multicolumn{3}{c}{raw material}\\
    \midrule
    Property&$A$&$B$&$C$ \\
    \midrule
    Specific gravity&0.92&0.97&1.04 \\
    Chromium \%&7&13&16 \\
    Melting point °C&440&490&480\\
    \bottomrule
  \end{tabular}
  }
%\includegraphics<1,2>[scale=0.5]{example_blending_gupta}
\par}

\only<2>{Los costos de las diversas materias primas por tonelada son: \$ 90 para A, \$ 280 para B y \$ 40 para C. Formule el modelo L.P. para encontrar las proporciones en las que se utilizarán A, B y C para obtener una aleación de las propiedades deseadas, mientras que el costo de las materias primas es mínimo.}
    
\end{frameExample}



%%% Local Variables:
%%% mode: latex
%%% TeX-master: "slides_linear-programming-intro"
%%% End:
