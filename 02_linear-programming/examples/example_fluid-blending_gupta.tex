\begin{frameExample}{Fluid Blending Problem}{}
  % Example 2.6-10 Gupta Ebook
  \only<1>{%
Una compañía petrolera produce dos grados de gasolina, $P$ y $Q$ que vende a \$ 30 y \$ 40 por litro. La empresa puede comprar cuatro crudos diferentes con los siguientes componentes y costos:

  {\centering
    \begin{tabular}{ccccc}
      \toprule
      Crudo&\multicolumn{3}{c}{Componente}& Precio / litro\\
      \cmidrule{2-4}
      oil&$A$&$B$&$C$& \$ \\
      \midrule
      1&0.75&0.15&0.10 & 20.00\\
      2&0.20&0.30&0.50&22.50\\
      3&0.70&0.10&0.20&25.00\\
      4&0.40&0.10&0.50&27.50\\
      \bottomrule
    \end{tabular}
    \par}

La gasolina $P$ debe tener al menos el 55 por ciento del componente $A$ y no más del 40 por ciento de $C$. La gasolina $Q$ no debe tener más del 25 por ciento de $C$. Determine cómo se deben usar los crudos para maximizar la utilidad.
  }

  \only<2>{%
    La decisión consiste en determinar la cantidad de crudo que se va a usar para cada tipo de gasolina. Las cantidades en litros se representan por la variable $x_{ij}$, en donde $i = \text{crudo } 1, 2, 3, 4$ y $j = \text{gasolina tipo } 1 \text{ y } 2$ respectivamente. Así tenemos que %

    {\centering
\begin{tabular}{r@{ : }l}
  $x_{11}$ & cantidad en litros del crudo $1$ usado en gasolina tipo $P$\\
  $x_{21}$ & cantidad en litros del crudo $2$ usado en gasolina tipo $P$\\
  $\cdots$ & $\quad \quad \cdots$ $\quad \quad \cdots$ $\quad \quad \cdots$ $\quad \quad \cdots$\\
  $\cdots$ & $\quad \quad \cdots$ $\quad \quad \cdots$ $\quad \quad \cdots$ $\quad \quad \cdots$\\
  $x_{12}$ & cantidad en litros del crudo $1$ usado en gasolina tipo $Q$\\
  $x_{22}$ & cantidad en litros del crudo $2$ usado en gasolina tipo $Q$\\
  $\cdots$ & $\cdots$ $\quad \quad \cdots$ $\quad \quad \cdots$ $\quad \quad \cdots$ $\quad \quad \cdots$\\
  $\cdots$ & $\cdots$ $\quad \quad \cdots$ $\quad \quad \cdots$ $\quad \quad \cdots$ $\quad \quad \cdots$
    \end{tabular}
    \par}
}%
\only<3>{  El objetivo es maximizar la ganancia neta \[\max Z = 30\sum_{i=1}^{4}x_{i1} + 40\sum_{i=1}^{4}x_{i2} - 20\sum_{j=1}^{2}x_{1j} -  22.50\sum_{j=1}^{2}x_{2j} - 25\sum_{j=1}^{2}x_{3j} - 27.50\sum_{j=1}^{2}x_{4j}\]
  sujeto a

  \begin{align*}
    0.75x_{11} + 0.20x_{21} + 0.75x_{31} + 0.40x_{41} & \geq 0.55(x_{11} + x_{21} + x_{31} + x_{41} ) \\
    0.10x_{11} + 0.50x_{21} + 0.20x_{31} + 0.50x_{41} & \leq 0.40(x_{11} + x_{21} + x_{31} + x_{41} ) \\
    0.10x_{12} + 0.50x_{22} + 0.20x_{32} + 0.50x_{42} & \leq 0.25(x_{12} + x_{22} + x_{32} + x_{42} ) \\[5mm]
    x_{11} , x_{21} , x_{31},x_{41}, x_{12} , x_{22} , x_{32},x_{42} & \geq 0
  \end{align*}
}
\end{frameExample}


%%% Local Variables:
%%% mode: latex
%%% TeX-master: "../slides"
%%% End:
