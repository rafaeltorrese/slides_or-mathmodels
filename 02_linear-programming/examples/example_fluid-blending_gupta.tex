\begin{frameExample}{10. Fluid Blending Problem}{}
  An oil cmpany produces tow grades of gasoline P and Q which it sells at \$ 30, and \$ 40 per litre. The company can buy four different crude oils withthe following constituents and costs:

  {\centering
    \begin{tabular}{ccccc}
      \toprule
      Crude&\multicolumn{3}{c}{Constituents}& Price / litre\\
      \cmidrule{2-4}
      oil&$A$&$B$&$C$& \$ \\
      \midrule
      1&0.75&0.15&0.10 & 20.00\\
      2&0.20&0.30&0.50&22.50\\
      3&0.70&0.10&0.20&25.00\\
      4&0.40&0.10&0.50&27.50\\
      \bottomrule
    \end{tabular}
    \par}

  Gasoline $P$ must have at least 55 per cent of constituent $A$ and no more than 40 per cent of $C$. Gasoline $Q$ must not have more than 25 per cent of $C$. Determine how the crudes should be used to maximize the profit.
\end{frameExample}


%%% Local Variables:
%%% mode: latex
%%% TeX-master: "../slides_linear-programming-intro"
%%% End:
