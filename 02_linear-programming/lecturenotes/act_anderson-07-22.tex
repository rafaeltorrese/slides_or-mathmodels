\begin{frameact}{Riser Sports Products}{}
  % Anderson 07-22
  Reiser Sports Products quiere determinar la cantidad de balones de futbol de All-Pro (A)
y Universitario (U) a producir con el fin de maximizar las utilidades durante el siguiente horizonte de planeación de cuatro semanas. Las restricciones que afectan las cantidades de
producción son las capacidades de producción en tres departamentos: corte y teñido, costura
e inspección y empaque. Para el periodo de planeación de cuatro semanas se dispone de 340 horas de corte y teñido, 420 horas de costura y 200 horas de inspección y empaque. Los tiempos requeridos para elaborar un balón A y U en el departamento de corte y teñido son de 12 y 6 horas respectivamente. El departamento de costura requiere 9 horas para un balón A y 6 horas para elaborar un balón U. En la inspección se ocupan 6 horas para cada balón de fútbol. Los balones de futbol All-Pro producen utilidades de \$5 por unidad y los balones Universitarios producen una utilidad de \$4 por unidad. Formule un modelo para el problema.
\end{frameact}



%%% Local Variables:
%%% mode: latex
%%% TeX-master: "../slides"
%%% End:
