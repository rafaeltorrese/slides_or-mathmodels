\question
  % Taha 02-02A-04
\label{act:taha_02-02A-04}
  Una compañía que funciona 10 horas al día fabrica dos productos en tres procesos secuenciales. La siguiente tabla resume los datos del problema:

  {
    \centering
    \begin{tabular}{ccccc}
      \toprule
      &\multicolumn{3}{c}{Minutos por unidad}&Utilidad\\
      \cmidrule{2-4}
      Producto& Proceso 1& Proceso 2& Proceso 3& Unitaria (\$)\\
      \midrule
1&10&6&8&2\\
      2&5&20&10&3\\
      \bottomrule
    \end{tabular}
    \par
  }
  
  Formule un modelo de Programación Lineal

  \begin{solution}
  
    {
      \centering
      $\max Z = 40x_1 + 35x_2$
\sysalign{r,r}%
    \systeme[x_1x_2]%
    {
      10x_1 + 5x_2 \leq 600,
      6x_1 + 20x_2 \leq 600,
      8x_1 + 10x_2 \leq 600
    }
    
    \vspace{5mm}
    
    $x_1, x_2 \geq 0$
      \par
    }
    
  \end{solution}

%%% Local Variables:
%%% mode: latex
%%% TeX-master: "activities"
%%% End:
