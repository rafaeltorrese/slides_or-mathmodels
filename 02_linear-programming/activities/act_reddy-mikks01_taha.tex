\begin{frameact}{Reddy Mikks }

\label{act:reddy-mikks}
  % Ejemplo 2.1-1 (La compañía Reddy Mikks) Taha
  \only<1>{%
    Reddy Mikks produce pinturas para interiores y exteriores con dos materias primas, $M_1$ y $M_2$. La tabla siguiente proporciona los datos básicos del problema. Una encuesta de mercado indica que la demanda diaria de pintura para interiores no puede exceder la de pintura para exteriores en más de una tonelada. Asimismo, que la demanda diaria máxima de pintura para interiores es de dos toneladas. Reddy Mikks se propone determinar la (mejor) combinación óptima de pinturas para interiores y exteriores que maximice la utilidad diaria total.%
  }


  {\centering
  \includegraphics<1>[scale=0.6]{reddy-mikks_01}
  \par}
\end{frameact}

\begin{frameact}{Reddy Mikks Formulations}{}

\label{act:taha_02-01A-01}
  Para el modelo de \hyperlink{act:reddy-mikks}{Reddy Mikks} , defina las siguientes restricciones y expréselas con un lado izquierdo lineal y un lado derecho constante

(a) La demanda diaria de pintura para interiores supera la de pintura para exteriores
por al menos una tonelada.

(b) El consumo diario de materia prima $M_2$ en toneladas es cuando mucho de 6 y por
lo menos de 3.

(c) La demanda de pintura para interiores no puede ser menor que la demanda de pintura para exteriores.

(d) La cantidad mínima de pintura que debe producirse tanto para interiores como para exteriores es de 3 toneladas.

(e) La proporción de pintura para interiores respecto de la producción total de pintura
para interiores y exteriores no debe exceder de 0.5

\end{frameact}




%%% Local Variables:
%%% mode: latex
%%% TeX-master: "../slides_linear-programming-intro"
%%% End:
