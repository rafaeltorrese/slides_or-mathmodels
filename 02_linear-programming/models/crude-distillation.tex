\documentclass[../../main.tex]{subfiles}

\begin{document}
\begin{frame}{Destilación de Crudos}{}
  \only<1>{%
    Una compañía de petróleos produce en sus refinerías gasóleo (C), gasolina sin plomo (P) y gasolina súper (S) a partir de dos tipos de crudos,$C_1$ y $C_2$. Las refinerías están dotadas de dos tipos de tecnologías.

    La tecnología nueva $T_n$ utiliza en cada sesión de destilación 7 unidades de $C_1$ y 12 de $C_2$, para producir 8 unidades de $G$, 6 de $P$ y 5 de $S$. Con la tecnología antigua $T_n$, se obtienen en cada destilación 10 unidades de $G$, 7 de $P$ y 4 de $S$, con un gasto de 10 unidades de $C_1$ y 8 de $C_2$.

  Estudios de demanda permiten estimar que para el próximo mes se deben producir al menos 900 unidades de $G$, 300 de $P$ y entre 800 y 1700 de $S$. La disponibilidad de crudo $C_1$ es de 1400 unidades y de $C_2$ de 2000 unidades.
  }

  \only<2>{%
    Los beneficios por unidad producida son


{    \centering
    \begin{tabular}{llll}
\toprule
        Gasolina & $G$ & $P$& $S$ \\ 
      Beneficio/u & 4 & 6 & 7 \\
      \bottomrule
    \end{tabular} \par}

  La compañía desea conocer cómo utilizar ambos procesos de destilación, que se pueden realizar total o parcialmente, y los crudos para que el beneficio sea el máximo.
  }
\end{frame}

\end{document}