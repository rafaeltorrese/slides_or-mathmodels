\documentclass[../../main.tex]{subfiles}
\captionsetup[table]{name=Tabla}
\begin{document}

\begin{frame}{Planificación de mezclas en una planta química}{}
  Una planta química fabrica dos productos $A$, $B$ mediante dos procesos \texttt{I} y \texttt{II}. La tabla~\ref{tab:processing-times} da los tiempos de producción de $A$ y $B$ en cada proceso y los beneficios (en miles de \$) por unidad vendida

  
  \begin{table}
    \caption{Tiempos de producción para los procesos.\label{tab:processing-times}}
    \centering
    \begin{tabular}[h]{crr}
      \toprule
      &\multicolumn{2}{c}{Producto}\\
      \cmidrule{2-3}
      Proceso & $A$&$B$\\
      \midrule
      \texttt{I}&2&3\\
      \texttt{II}&3&4\\
      Beneficio/u&4&10\\
      \bottomrule      
    \end{tabular}        
  \end{table}    
\end{frame}

\begin{frame}{Planificación de mezclas en una planta química}{}
  Se dispone de 16 horas de operación del proceso \texttt{I} y de 24 horas del \texttt{II}. La producción de $B$ da, además, un subproducto $C$ (sin coste adicional) que puede venderse a 3000 \$/u. Sin embargo, el sobrante de $C$ debe destruirse con coste 2000 \$/u. Se obtienen 2 unidades de $C$ por cada unidad de $B$ producida. La demanda de $C$ se estima en, a lo mucho, 5 unidades. Formular un programa lineal que dé el plan de producción con máximo beneficio.
\end{frame}
\end{document}