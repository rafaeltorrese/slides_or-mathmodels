\documentclass[../../main.tex]{subfiles}
\captionsetup[table]{name=Tabla}
\begin{document}
\begin{frame}[t]{Optimización de mezclas en una destilería.}{}
  \only<1>{%
    Una destilería dispone de malta propia en cantidad de 200 barriles/día. Además, puede comprar malta de dos distribuidores $A$ y $B$, con costes de 1000 y 1200 \$/barril, en cantidades máximas de 300 y 500 barriles/día, respectivamente.

    La malta puede mezclarse directamente o destilarse para producir malta enriquecida de dos tipos 1, 2. El destilador puede procesar a lo sumo 700 barriles/día. Un barril destilado de la propia casa produce 0.3 barriles de malta 1 y 0.6 de malta 2. Un barril de malta $A$ produce 0.4 de 1 y 0.4 de 2. Uno de malta $B$ produce 0.7 de 1 y 0.1 de 2. La mezcla de malta no procesada se vende a 1300 \$/barril, limitándose el mercado a 110 barriles/día.  El sobrante de malta debe destruirse con coste 100 \$/barril.

    Con las maltas destiladas pueden hacerse dos productos: uno de alta calidad ($H$), que se vende a 1900 \$/barril y debe contener al menos el 70\% de producto 1, y otro de baja calidad ($L$), que se vende a 1500 \$/barril y puede contener a lo sumo el 55\% de producto 2.%
  }

\only<2>{%  
  La destilería desea satisfacer la demanda del producto de alta calidad, que es de 215 barriles/día, y asegurarse un beneficio de 30,000 \$/día. Además, puesto que se espera un cambio en el mercado del producto de baja calidad, la destilería desea minimizar su producción.

  Formular un modelo de programación lineal que de respuesta al problema de planificación planteado teniendo en cuenta las limitaciones en la producción y las exigencias de demanda y beneficio económico, suponiendo, además, que la venta de la mezcla está garantizada.
  }

\end{frame}
\end{document}