\section{Variables Artificiales}
\label{sec:artificial-variables}


\begin{frame}{Técnicas de Variable Artificial}
  \begin{itemize} \parskip3mm \justifying
  \item<only@1> En los problemas anteriores, las restricciones eran de tipo ($\leq$) (con lados derechos no negativos). La introducción de variables de holgura proporcionó fácilmente la solución factible básica inicial. 
  \item<only@1> Existen  problemas de programación lineal en los que las \alert{variables de holgura no pueden proporcionar una solución factible básica inicial.} 
  \item<only@1> En estos problemas, \alert{al menos una de las restricciones es de tipo ($\geq$) o ($=$)}. En tales casos, presentamos otro tipo de variables llamadas \alert{variables artificiales}. Estas variables son ficticias y no tienen significado físico. 
  \item<only@1> Asumen el papel de \alert{variables de holgura en la primera iteración}, solo para ser \alert{reemplazadas en una iteración posterior. }
  \item<only@2> Método de Penalización.
  \item<only@2> Método Dos Fases.   
  \end{itemize}
\end{frame}



\begin{frame}{Método de Penalización. Ejemplo}{}

  \begin{onlyenv}<1>
      \begin{columns}
    \column{0.5\textwidth}
  \begin{align*}
    \max Z = 2x_1 + 3x_2 & \\[5mm]
    x_1 + x_2 & \leq 30\\
    x_2 & \geq 3\\
    x_2 & \leq 12\\
    x_1 - x_2 & \geq 0\\
    x_1  & \leq 20 \\[5mm]
    x_1, x_2 & \geq 0
  \end{align*}
  \column{0.5\textwidth}
  \begin{align*}
   \max Z = 2x_1 + 3x_2 - MA_2& \\[5mm]
    x_1 + x_2 + s_1  & = 30\\
    x_2  - s_2  + A_2 &= 3\\
    x_2  + s_3 &= 12\\
    -x_1 + x_2 + s_4 & =  0\\
    x_1  + s_5& = 20 \\[5mm]
    x_1, x_2, s_1, s_2, s_3, s_4, s_5, A_2 & \geq 0
  \end{align*}
  \end{columns}
\end{onlyenv}

\begin{onlyenv}<2>

  \[ \max Z = 2x_1 + 3x_2 - MA_2 \]
      \begin{align*}
    x_1 + x_2 + s_1  & = 30\\
    x_2  - s_2  + A_2 &= 3\\
    x_2  + s_3 &= 12\\
    -x_1 + x_2 + s_4 & =  0\\
    x_1  + s_5& = 20 \\[5mm]
    x_1, x_2, s_1, s_2, s_3, s_4, s_5, A_2 & \geq 0
  \end{align*}
\end{onlyenv}
\end{frame}


\begin{frame}{Método de las Dos Fases. Ejemplo}{}

  \begin{block}{Fase I} \justifying
  \begin{itemize} \parskip3mm \justifying
  \item Todas las variables artificiales se hacen cero. Esto se hace con una función objetivo artificial en la que se expresa la suma de variables artificiales.
  \item La nueva \alert{función objetivo se minimiza}, sujeta a las restricciones del problema original.
  \end{itemize}
\end{block}
\begin{block}{Fase II}\justifying
  La tabla final de la Fase I es la tabla inicial de la Fase II. Las variables artificiales en la función objetivo se reemplazan  por las variables originales y se aplica el método simplex.
\end{block}
\end{frame}    

\begin{frame}{Método de las Dos Fases. Ejemplo}{}
  \begin{columns}
    \column{0.5\textwidth}
  \begin{align*}
    \max Z = 2x_1 + 3x_2 & \\[5mm]
    x_1 + x_2 & \leq 30\\
    x_2 & \geq 3\\
    x_2 & \leq 12\\
    -x_1 + x_2 & \leq 0\\
    x_1  & \leq 20 \\[5mm]
    x_1, x_2 & \geq 0
  \end{align*}  
  \column{0.5\textwidth}
  \begin{align*}
    \min W = A2 & \\[5mm]
    x_1 + x_2 + s_1  & = 30\\
    x_2  - s_2  + A_2 &= 3\\
    x_2  + s_3 &= 12\\
    -x_1 + x_2 + s_4 & =  0\\
    x_1  + s_5& = 20 \\[5mm]
    x_1, x_2, s_1, s_2, s_3, s_4, s_5, A_2 & \geq 0
  \end{align*} 
  \end{columns}
\end{frame}

%%% Local Variables:
%%% mode: latex
%%% TeX-master: "slides_simplex"
%%% End:
