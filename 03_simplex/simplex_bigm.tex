
\section{Variables Artificiales}
\label{sec:artificial-variables}


\begin{frame}{Técnicas de Variable Artificial}
  \begin{itemize} \parskip3mm \justifying
  \item<only@1> En los problemas anteriores, las restricciones eran de tipo ($\leq$) (con lados derechos no negativos). La introducción de variables de holgura proporcionó fácilmente la solución factible básica inicial. 
  \item<only@1> Existen  problemas de programación lineal en los que las \alert{variables de holgura no pueden proporcionar una solución factible básica inicial.} 
  \item<only@1> En estos problemas, \alert{al menos una de las restricciones es de tipo ($\geq$) o ($=$)}. En tales casos, presentamos otro tipo de variables llamadas \alert{variables artificiales}. Estas variables son ficticias y no tienen significado físico. 
  \item<only@1> Asumen el papel de \alert{variables de holgura en la primera iteración}, solo para ser \alert{reemplazadas en una iteración posterior. }
  \item<only@2> Método de Penalización.
  \item<only@2> Método Dos Fases.   
  \end{itemize}
\end{frame}

\begin{frame}{Método de Penalización. Ejemplo}{}
  \begin{align*}
    \max Z = 2x_1 + 3x_2 & \\[5mm]
    x_1 + x_2 & \leq 30\\
    x_2 & \geq 3\\
    x_2 & \leq 12\\
    x_1 - x_2 & \geq 0\\
    x_1  & \leq 20 \\[5mm]
    x_1, x_2 & \geq 0
  \end{align*}  
\end{frame}


%%% Local Variables:
%%% mode: latex
%%% TeX-master: "slides_simplex"
%%% End:
