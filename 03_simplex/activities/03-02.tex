\documentclass[spanish,letterpaper,11pt]{exam}


% -------------------- Packages --------------------
\usepackage{amsmath}
\usepackage{amssymb}
\usepackage{booktabs}
\usepackage{systeme}
\usepackage{cancel}
% -------------------- Definitions --------------------

\newcommand{\tema}{Método Analítico}
%
\extrafootheight{-0.5in}
\header%
{Dr. Rafael Torres Escobar}% Left
{INVESTIGACIÓN DE OPERACIONES -- MODELOS MATEMÁTICOS \\ \tema}% Center
{Página \\ \thepage\ de \numpages} % Right
\headrule
\pointpoints{punto}{puntos}
\author{Dr. Rafael Torres Escobar}
%
% \printanswers % comentar para mostrar respuestas
% ================================================== 
\begin{document}
\begin{questions}
    \question
    Resolver por método analítico. Mostrar todos los sistemas de ecuaciones. Clasificar cada solución como básica, no-básica, factible e infactible según corresponda.

\[ \max Z = 2x_1 + 3x_2\]
{\centering
  subject to

  \sysalign{r,r}%
  \sysdelim..%
  \systeme[x_1x_2]%
  {
    x_1 + x_2 \leq 30,
    x_2 \geq 3,
    x_2 \leq 12,
    x_1 - x_2 \geq 0,
    x_1 \leq 20
  }

  $x_1, x_2 \geq 0$
  \par}

    \begin{solution}
        
        Standard form: 

        \[ Z = 2x_1 + 3x_2\]
        {\centering
          subject to
        
          \sysalign{r,r}%
          \sysdelim..%
          \systeme[x_1x_2s_1s_2s_3_s_4s_5]%
          {
            x_1 + x_2 + s_1 = 30,
            x_2 - s_2 = 3,
            x_2 + s_3 = 12,
            x_1 - x_2 - s_4 = 0,
            x_1 + s_5 = 20
          }
        
          $x_1, x_2, s_1, s_2, s_3, s_4, s_5 \geq 0$
          \par}
          
          \begin{flalign*}
              m & =  5\\
              m + n & = 7\\
              _{(m + n)}C_{m} & = \frac{7!}{5!2!} = \frac{7 \cdot 6}{2!} = 21
          \end{flalign*}
       Total number of  combinations: 
    \end{solution}

    \vspace{1cm}
    \question
    % EXAMPLE 2.6-1 (Production Allocation Problem} Gupta ebook
  Una empresa produce tres productos. Estos productos se procesan en tres máquinas diferentes. El tiempo requerido para fabricar una unidad de cada uno de los tres productos y la capacidad diaria de las tres máquinas se detallan en la tabla a continuación.

  {\centering
      \begin{tabular}{ccccc}
        \toprule
        Máquina & \multicolumn{3}{l}{Tiempo por unidad (minutos)} & Capacidad       \\
                &  Producto 1             &    Producto 2            &     Producto 3           & (Minutos / día) \\
        \midrule
        $M_1$   & 2             & 3              & 2              & 440             \\
        $M_2$   & 4             & --             & 3              & 470             \\
        $M_3$   & 2             & 5              & --             & 430\\
        \bottomrule
      \end{tabular}
    \par}
  

   Se requiere determinar la cantidad diaria de unidades que se fabricarán para cada producto. El beneficio por unidad para el producto 1, 2 y 3 es de \$ 4, \$ 3 y \$ 6 respectivamente. Se supone que todas las cantidades producidas se consumen en el mercado. Formule el modelo matemático (L.P.) que maximizará la ganancia diaria. 
   
   \begin{itemize}
       \item Resolver por método analítico. 
       \item Expresar todos los sistemas de ecuaciones que intervienen. 
       \item Clasificar soluciones como básicas, no básicas y factibles e infactibles.
   \end{itemize}

   \begin{solution}
       
    Standard form: 

        \[ \max Z = 4x_1 + 3x_2 + 6x_3 + 0s_1 + 0s_2 + 0s_3\]
        {\centering
          subject to
        
          \sysalign{r,r}%
          \sysdelim..%
          \systeme[x_1x_2x_3s_1s_2s_3]%
          {
            2x_1 + 3x_2 + 2x_3 +  s_1 = 440,
            4x_1 +      + 3x_3 +  s_2 = 470,
            2x_1 + 5x_2        +  s_3 = 430,
          }        
          $x_1, x_2, x_3, s_1, s_2, s_3 \geq 0$
          \par}
          
          \begin{flalign*}
              m & =  3\\
              m + n & = 6\\
              _{(m + n)}C_{m} & = \frac{6!}{3!3!} = \frac{6 \cdot 5 \cdot 4 \cdot}{3!} = 20
          \end{flalign*}
       
   \end{solution}
\end{questions}
\end{document}