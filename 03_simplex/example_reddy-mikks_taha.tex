\begin{frameExample}{Reddy Mikks  }

  \label{example:reddy-mikks}
  \begin{onlyenv}<1>
      % Ejemplo 2.1-1 (La compañía Reddy Mikks) Taha
    Reddy Mikks produce pinturas para interiores y exteriores con dos materias primas, $M_1$ y $M_2$. La tabla siguiente proporciona los datos básicos del problema. Una encuesta de mercado indica que la demanda diaria de pintura para interiores no puede exceder la de pintura para exteriores en más de una tonelada. Asimismo, que la demanda diaria máxima de pintura para interiores es de dos toneladas. Reddy Mikks se propone determinar la (mejor) combinación óptima de pinturas para interiores y exteriores que maximice la utilidad diaria total.%

    {\centering
      \scalebox{0.7}{%
        \begin{tabular}{rcccc}
          \toprule
          & \multicolumn{2}{l}{Toneladas de materia prima por tonelada de} &  &  \\
          \cmidrule{2-3}
          & Pintura para & Pintura para & Disponibilidad diaria &  \\
          Materia Prima & interiores & exteriores & máxima (toneladas) &  \\
          \midrule
          $M1$ & 6 & 4 & 24 &  \\
          $M2$ & 1 & 2 & 6 &  \\
          \midrule
          Utilidad por &  &  &  &  \\
          tonelada (\$1000) & 5 & 4 &  & \\
          \bottomrule
        \end{tabular}
      } % scalebox
      \par}
  \end{onlyenv}
  
\begin{block}<only@2>{Reddy Mikks (Modelo) }\justifying

\[\max Z = 5x_1 + 4x_2\]

{\centering
  subject to

  \sysdelim..\sysalign{r,r}\systeme[x_1x_2]%
  {
    6x_1 + 4x_2  \leq 24,
    x_1 + 2x_2  \leq 6,
    -x_1 + x_2  \leq 1,
    x_2  \leq 2
  }

  \vspace{5mm}
  $x_1, x_2  \geq 0$
  \par}
\end{block}
\end{frameExample}




%%% Local Variables:
%%% mode: latex
%%% TeX-master: "slides_simplex"
%%% End:
