% !TeX root = slides.tex

\section{Canonical and Standard Forms of L.P.P.}
\label{sec:canonical-standard-form}

\begin{frame}{Forma Canónica y Estándar}
  \begin{columns}
    \column{0.5\textwidth}
      \begin{block}{Forma Canónica}
  Maximizar \[ Z = \sum_{j=1}^{n} c_j x_j\] 
  sujeto a
  \begin{align*}
    \sum a_{ij}x_j  & \leq b_i, \quad i = 1, 2, \ldots, m,\\
    x_j  & \geq 0, \quad j = 1, 2, \ldots, n,\\
  \end{align*}  
\end{block}
\column{0.5\textwidth}
\begin{block}{Forma Estándar}
    Maximizar \[ Z = \sum_{j=1}^{n} c_j x_j\] 
  sujeto a
  \begin{align*}
    \sum a_{ij}x_j  & = b_i, \quad i = 1, 2, \ldots, m,\\
    x_j  & \geq 0, \quad j = 1, 2, \ldots, n,\\
  \end{align*}
\end{block}
  \end{columns}
\end{frame}


\begin{frame}{Forma Estándar Para P.P.L.}
  \begin{onlyenv}<1>
    
    Para resolver un Problema de Programación Lineal (P.P.L.), éste se debe expresar en la forma estándar.\\

    Maximizar \[ Z = \sum_{j=1}^{n} c_j x_j\] 
  sujeto a
  \begin{align*}
    \sum a_{ij}x_j  & \leq  b_i, \, (\geq b_i), & i = 1, 2, \ldots, m,\\
    x_j  & \geq 0, & j = 1, 2, \ldots, n,\\
  \end{align*}
\end{onlyenv}
\begin{onlyenv}<2>
  Expresado en forma estándar obtenemos\\

  Maximizar \[ Z = \sum_{j=1}^{n} c_j x_j\] 
  sujeto a
  \begin{align*}
    \sum a_{ij}x_j + s_i & =  b_i,  \,\, i = 1, 2, \ldots, m,\\
    x_j  & \geq 0, \, \, j = 1, 2, \ldots, n,\\
    s_i  & \geq 0, \, \, i = 1, 2, \ldots, m.\\
  \end{align*}
\end{onlyenv}
\end{frame}


\begin{frameExample}{Standard Form \label{example:02.12-01}}{}

\[     \max Z = 7x_1 + 5x_2 \]

{\centering
  subject to
  
  \sysdelim..%
  \sysalign{r,r}%
  \systeme[x_1x_2]%
  {
    2x_1 + 3x_2  \leq 20,
    3x_1 + x_2 \geq 10
  }

    \vspace{3mm}
    $x_1, x_2  \geq 0$
    \par}





\end{frameExample}

\begin{frameExample}{Standard Form \label{example:02.12-02}}{}
      % EXAMPLE  Gupta ebook 2.12-2
    \[ \max Z = 3x_1 + 2x_2 + 5x_3\]
    {\centering
      subject to

      \sysdelim..%
      \sysalign{r,r}%
      \systeme[x_1x_2x_3]%
      {
        2x_1 - 3x_2  \leq 3,
        x_1 + 2x_2 + 3x_3 \geq 5,
        3x_1 + 2x_3 \leq 2
      }
\vspace{3mm}

    $x_1, x_2  \geq 0$
    \par}
    
  \end{frameExample}
  

\begin{frameExample}{Unrestricted variables \label{example:02-12-03}}{}

      % Example Gupta ebook 2.12-3 
    \[\max Z = 3x_1 + 2x_2 + 5x_3 \]
    {\centering
      subject to

      \sysdelim..%
      \sysalign{r,r}%
      \systeme[x_1x_2x_3]%
      {
        2x_1 + 3x_2 - 2x_3 \leq 40,
        4x_1 - 2x_2 + x_3\leq 24,
        x_1 - 5x_2 - 6x_3\geq 2
      }

\vspace{3mm}
    $    x_1  \geq 0$
        \par}
\end{frameExample}

\begin{frameExample}{Unrestricted variables \label{example:02.12-04}}{}
  % Gupta ebook Example 02.12-04
  \[ \max Z = 2x_1 + 3x_2 \]
  {\centering
    subject to
    
    \sysdelim..%
    \sysalign{r,r}%
    \systeme[x_1x_2x_3x_4x_5x_6]%
    {
      2x_1 - 3x_2 - x_3 = -4,
    3x_1 + 4x_2 - x_4 = -6,
    2x_1 + 5x_2 + x_5 = 10,
    4x_1 - 3x_2 + x_6 = 18
    }

    \vspace{3mm}
    $x_3, x_4,x_5, x_6  \geq 0$
  \par}
\end{frameExample}

\begin{frameExample}{Unrestricted variables \label{example:02.12-05}}{}
  % Example 2.12-5
  
    \[ \max Z = 3x_1 + 5x_2 - 2x_3\]
    {\centering
      subject to

      \sysdelim..%
      \sysalign{r,r}%
      \systeme[x_1x_2x_3]%
      {
        x_1 + 2x_2 - x_3 \geq -4,
        -5x_1 + 6x_2 +7x_3 \geq 5,
        2x_1 + x_2 + 3x_3 = 10
      }
    \par}

  \vspace{3mm}
  $    x_1, x_2 \geq 0,$
    $x_3  \text{ unrestricted}$
\end{frameExample}

\begin{frameExample}{Matrix Form\label{example:02.12-06}}{}
        % Example 2.12-6
        
    \[\max Z = 4x_1 + 2x_2 + 6x_3 \]

    {\centering
      subject to

      \sysdelim..%
      \sysalign{r,r}%
      \systeme[x_1x_2x_3]%
      {
        2x_1 + 3x_2 + 2x_3 \geq 6,
        3x_1 + 4x_2 = 8,
        6x_1 - 4x_2 + x_3\leq 10
      }

      \vspace{3mm}
      $x_1, x_2,x_3  \geq 0$
    \par}

\end{frameExample}



%%% Local Variables:
%%% mode: latex
%%% TeX-master: "slides"
%%% End:
