\begin{frameExample}{Trial and Error Method \label{example:02.15-03}}{}
  % EXAMPLE  2.15-3 Gupta Ebook
  \begin{onlyenv}<1>
      Una empresa fabrica cuatro partes diferentes de una máquina $M_1, M_2,$ $ M_3, M_4$ con una aleación de cobre y zinc. Las cantidades de cobre y zinc requeridas para cada parte, su \alert{disponibilidad exacta} y la ganancia de cada unidad para cada parte es la siguiente

  {\centering
    \begin{tabular}{rccccc}
      \toprule
      &&&&&Disponibilidad\\
      &$M_1$&$M_2$&$M_3$&$M_4$&exacta\\
      &(kg)&(kg)&(kg)&(kg)&(kg)\\
      \midrule
      Cobre&5&4&2&1&100\\
      Zinc&2&3&8&1&75\\
      Ganancia&12&8&14&10&\\
      \bottomrule
    \end{tabular}
  \par}

¿Cuántas partes se deben fabricar para maximizar la ganancia?
\end{onlyenv}
\begin{onlyenv}<2>
  \[ \max Z = 12x_1 + 8x_2 + 14x_3 + 10x_4\]

  {
    \centering
    subject to

    \sysdelim..%
    \sysalign{r,r}%
    \systeme[x_1x_2x_3x_4]%
    {
      5x_1 + 4x_2 + 2x_3 + x_4 = 100,
      2x_1 + 3x_2 + 8x_3 + x_4 = 75
    }

    \vspace{4mm}
    $x_1, x_2,x_3, x_4 \geq 0$
    \par
  }
\end{onlyenv}
\end{frameExample}

%%% Local Variables:
%%% mode: latex
%%% TeX-master: "slides_simplex"
%%% End:
