
\section{Método Simplex}
\label{sec:simplex-method}


\begin{frame}{Algoritmo Simplex}
  \begin{itemize} \parskip3mm \justifying
  \item<only@1>   El método gráfico no se puede aplicar cuando el número de variables involucradas en el problema de LP es más de tres o más bien dos, ya que incluso \alert{con tres variables la solución gráfica se vuelve tediosa} ya que involucra la intersección de planos en tres dimensiones.

  \item<only@1>   El método simplex, desarrollado por el profesor George B. Dantzig, se puede utilizar para resolver cualquier problema de LP (para el que existe la solución) que involucre cualquier número de variables y restricciones (cientos o incluso miles).
  
  \item<only@2>   El procedimiento computacional en el método simplex \alert{se basa en la propiedad fundamental de que la solución óptima a un problema de LP, si existe, ocurre solo en uno de los puntos de esquina de la región factible.} El método simplex siempre comienza con la solución factible básica inicial, es decir, el origen, que es uno de los puntos de esquina de la región factible.

  \item<only@2>   A continuación, se prueba esta solución, es decir, \alert{se determina si es posible mejorar el valor de la función objetivo moviéndose al siguiente punto de esquina de la región factible.} Si es así, se obtiene la solución en este punto. Esta búsqueda del mejor punto de esquina se repite, hasta que después de un número finito de intentos, se obtiene la solución óptima, si existe.
  \end{itemize}
\end{frame}


\begin{frameExample}{09. Algoritmo Simplex}{}
  % EXAMPLE  2.9-3 Gupta ebook
  Resolver ejemplo 06 con algoritmo simplex
  \begin{columns}
    \column{0.5\textwidth}
        \begin{align*}
      \max Z = 2x_1 + x_2 & \\[3mm]
    x_1 + 2x_2 & \leq 10\\
    x_1 + x_2 & \leq 6\\
    x_1 - x_2 & \leq 2\\
    x_1 - 2x_2 & \leq 1\\[5mm]
    x_1, x_2 & \geq 0
        \end{align*}
        \column{0.5\textwidth}
        \[ \max Z = 2x_1 + x_2 \]
      Sujeto a %
      \scalebox{0.95}{%
\systeme[x_1x_2s_1s_2s_3s_4]{%
    x_1 + 2x_2 + s_1  = 10,
    x_1 + x_2 + s_2 = 6,
    x_1  - x_2  + s_3 = 2,
    x_1  - 2x_2  + s_4 = 1,
  } % systeme
}% scalebox

$x_1, x_2, s_1, s_2, s_3, s_4  \geq 0$
  \end{columns}
\end{frameExample}

\begin{frameExample}{09. Algoritmo Simplex}{}
  Tabla Simplex Inicial
  
  {\centering
\begin{tabular}{rc|rr|rrrr|r}
  &$\max$  & $x_1$ & $x_2$ & $s_1$ &$ s_2$ & $s_3$ & $s_4$ &  \\
  \toprule
  $\mathbf{C_b}$ & \textbf{basis} & 2 & 1 & 0 & 0 & 0 & 0 & RHS \\
  \midrule
0 & $s_1$ & 1 & 2 & 1 & 0 & 0 & 0 & 10 \\
0 & $s_2$ & 1 & 1 & 0 & 1 & 0 & 0 & 6 \\
0 & $s_3$ & 1 & -1 & 0 & 0 & 1 & 0 & 2 \\
  0 & $s_4$ & 1 & -2 & 0 & 0 & 0 & 1 & 1\\
  \bottomrule
\end{tabular}
  \par}
\end{frameExample}

\begin{frameExample}{09. Algoritmo Simplex}{}

  
  {\centering
      \begin{tabular}{rc|rrrrrr|rr}
  &  $\max$ & $x_1$ & $x_2$ & $s_1$ &$ s_2$ & $s_3$ & $s_4$ & & \\
  \toprule
$\mathbf{C_b}$ & \textbf{basis} & 2 & 1 & 0 & 0 & 0 & 0 & RHS & ratios \\
  \midrule
  \only<1>{
  0 & $s_1$ & 1 & 2 & 1 & 0 & 0 & 0 & 10 & 10 \\
0 & $s_2$ & 1 & 1 & 0 & 1 & 0 & 0 & 6 & 6 \\
0 & $s_3$ & 1 & -1 & 0 & 0 & 1 & 0 & 2 & 2 \\
  0 & $s_4$ & \cellcolor{yellow}1 & -2 & 0 & 0 & 0 & 1 & 1 & 1 \\
  \midrule
Iter& $Z_j$ & 0 & 0 & 0 & 0 & 0 & 0 & 0 &  \\
 1& $c_j - Z_j$ & 2 & 1 & 0 & 0 & 0 & 0 &  & 
                                             }% only
                                             \only<2>{
                                             0 & $s_1$ & 0 & 4 & 1 & 0 & 0 & -1 & 9 &  \\
0 & $s_2$ & 0 & 3 & 0 & 1 & 0 & -1 & 5 &  \\
0 & $s_3$ & 0 & 1 & 0 & 0 & 1 & -1 & 1 &  \\
        2 & $x_1$ & \cellcolor{yellow}1 & -2 & 0 & 0 & 0 & 1 & 1 &  \\
        \midrule
Iter & $Z_j$ & 2 & -4 & 0 & 0 & 0 & 2 & 2 &  \\
1 & $c_j - Z_j$ & 0 & 5 & 0 & 0 & 0 & -2 &  & 
                                              }%only
                                              \only<3>{%
0 & $s_1$ & 0 & 4 & 1 & 0 & 0 & -1 & 9 & 2.25 \\
0 & $s_2$ & 0 & 3 & 0 & 1 & 0 & -1 & 5 & 1.67 \\
0 & $s_3$ & 0 & \cellcolor{yellow}1 & 0 & 0 & 1 & -1 & 1 & 1.00 \\
        2 & $x_1$ & 1 & -2 & 0 & 0 & 0 & 1 & 1 & -0.50 \\
        \midrule
Iter & $Z_j$ & 2 & -4 & 0 & 0 & 0 & 2 & 2 &  \\
2 & $c_j - Z_j$ & 0 & 5 & 0 & 0 & 0 & -2 &  & 
                                              }%only
                                              \only<4>{%
0 & $s_1$ & 0 & 0 & 1 & 0 & -4 & 3 & 5 &  \\
0 & $s_2$ & 0 & 0 & 0 & 1 & -3 & 2 & 2 &  \\
1 & $x_2$ & 0 & \cellcolor{yellow}1 & 0 & 0 & 1 & -1 & 1 &  \\
        2 & $x_1$ & 1 & 0 & 0 & 0 & 2 & -1 & 3 &  \\
        \midrule
Iter & $Z_j$ & 2 & 1 & 0 & 0 & 5 & -3 & 7 &  \\
2 & $c_j - Z_j$ & 0 & 0 & 0 & 0 & -5 & 3 &  & 
                                              }% only
                                              \only<5>{%
0 & $s_1$ & 0 & 0 & 1 & 0 & -4 & 3 & 5 & 1.67 \\
0 & $s_2$ & 0 & 0 & 0 & 1 & -3 & \cellcolor{yellow}2 & 2 & 1 \\
1 & $x_2$ & 0 & 1 & 0 & 0 & 1 & -1 & 1 & -1 \\
        2 & $x_1$ & 1 & 0 & 0 & 0 & 2 & -1 & 3 & -3 \\
        \midrule
Iter & $Z_j$ & 2 & 1 & 0 & 0 & 5 & -3 & 7 &  \\
3 & $c_j - Z_j$ & 0 & 0 & 0 & 0 & -5 & 3 &  & 
                                              } % only
                                              \only<6>{%
0 & $s_1$ & 0 & 0 & 1 & -1.5 & 0.5 & 0 & 2 &  \\
0 & $s_4$ & 0 & 0 & 0 & 0.5 & -1.5 & \cellcolor{yellow}1 & 1 &  \\
1 & $x_2$ & 0 & 1 & 0 & 0.5 & -0.5 & 0 & 2 &  \\
        2 & $x_1$ & 1 & 0 & 0 & 0.5 & 0.5 & 0 & 4 &  \\
        \midrule
Iter & $Z_j$ & 2 & 1 & 0 & 1.5 & 0.5 & 0 & 10 &  \\
3 & $c_j - Z_j$ & 0 & 0 & 0 & -1.5 & -0.5 & 0 &\textbf{END}  & 
                                              } % only
\end{tabular}
  \par}
\end{frameExample}

%%% Local Variables:
%%% mode: latex
%%% TeX-master: "slides_simplex"
%%% End:

\begin{frameExample}{Reddy Mikks  }

\label{example:reddy-mikks}
  % Ejemplo 2.1-1 (La compañía Reddy Mikks) Taha
  \only<1>{%
    Reddy Mikks produce pinturas para interiores y exteriores con dos materias primas, $M_1$ y $M_2$. La tabla siguiente proporciona los datos básicos del problema. Una encuesta de mercado indica que la demanda diaria de pintura para interiores no puede exceder la de pintura para exteriores en más de una tonelada. Asimismo, que la demanda diaria máxima de pintura para interiores es de dos toneladas. Reddy Mikks se propone determinar la (mejor) combinación óptima de pinturas para interiores y exteriores que maximice la utilidad diaria total.%
  }


  {\centering
  \includegraphics<1>[scale=0.6]{reddy-mikks_01}
  \par}
\end{frameExample}




%%% Local Variables:
%%% mode: latex
%%% TeX-master: "../slides"
%%% End:


\begin{frameExample}{11. Producción}{}
  % EXAMPLE 2.6-1 (Production Allocation Problem} Gupta ebook
  Una empresa produce tres productos. Estos productos se procesan en tres máquinas diferentes. El tiempo requerido para fabricar una unidad de cada uno de los tres productos y la capacidad diaria de las tres máquinas se detallan en la tabla a continuación.

  {\centering
    \scalebox{0.8}{%
      \begin{tabular}{ccccc}
        \toprule
        Máquina & \multicolumn{3}{l}{Tiempo por unidad (minutos)} & Capacidad       \\
                &      Producto 1         &  Producto 2              &  Producto 3              & (Minutos / día) \\
        \midrule
        $M_1$   & 2             & 3              & 2              & 440             \\
        $M_2$   & 4             & --            & 3              & 470             \\
        $M_3$   & 2             & 5              & --             & 430\\
        \bottomrule
      \end{tabular}
    }% scalebox
    \par}
  
  

   Se requiere determinar la cantidad diaria de unidades que se fabricarán para cada producto. El beneficio por unidad para el producto 1, 2 y 3 es de \$ 4, \$ 3 y \$ 6 respectivamente. Se supone que todas las cantidades producidas se consumen en el mercado. Formule el modelo matemático (L.P.) que maximizará la ganancia diaria.
\end{frameExample}

\begin{frameExample}{11. Producción}{}


      \begin{align*}
      \max Z = 4x_1 + 3x_2 + 6x_3 & \\[3mm]
      \text{sujeto a } & \\[2mm]
    2x_1 + 3x_2 + 2x_3 & \leq 440\\
    4x_1 + 0x_2 + 3x_3 & \leq 470\\
    2x_1 + 5x_2  + 0x_3& \leq 430\\[5mm]
    x_1, x_2,x_3 & \geq 0
  \end{align*}
\end{frameExample}

%%% Local Variables:
%%% mode: latex
%%% TeX-master: "slides_simplex"
%%% End:
